\documentclass[12pt]{article}

%useful packages
\usepackage{color,soul}
\usepackage[usenames,dvipsnames,svgnames,table]{xcolor}
\usepackage{amsmath,amsthm,amscd,amssymb,bm}
\usepackage{hyperref}
\hypersetup{
    colorlinks=true,
    linkcolor=JungleGreen
}
\usepackage[utf8]{inputenc}
\usepackage[top=2cm, bottom=3cm, left=2cm, right=2cm]{geometry}
\usepackage{pgfplots}
\usepackage{enumitem}
\usepgfplotslibrary{fillbetween}
\usetikzlibrary{patterns}
\usepackage{tcolorbox}
\usepackage{centernot}
\usepackage{mathtools}
\usepackage{xcolor}

%personal definitions and commands
\newcommand{\R}{\mathbb{R}} 
\newcommand{\E}{\mathbb{E}}
\newcommand{\V}{\mathbb{V}}
\newcommand{\C}{\mathbb{C}}
\newcommand{\Prob}{\mathbb{P}}
\newcommand{\e}{\epsilon}
\newcommand\numberthis{\addtocounter{equation}{1}\tag{\theequation}} %allows numbering of single equations in align* environment
\newcommand{\mtx}[1]{\ensuremath{\bm{\mathit{#1}}}}
\newcommand{\B}{\hat{\boldsymbol{\beta}}}
\newcommand{\Cov}{\mathbb{C}\text{ov}}
\newcommand{\N}{\mathcal{N}}



\title{Notes on MNS (2016)}
\author{Anirudh Yadav}
\setlength\parindent{0pt}
\begin{document}

\maketitle

%\setcounter{tocdepth}{2}
%\tableofcontents


\section{Overview and results}
Main results:
\begin{enumerate}
\item In standard NKM there is an outsized response of output/consumption to forward guidance.
\item In a NKM with idiosyncratic labor income shocks and borrowing constraints, the response of output/consumption to forward guidance is far lower.
\item Something about ZLB [fill in]
\end{enumerate}

\subsection{Intuition [for my own understanding]}

\section{Result \#1}
I've replicated this in AIM. Some notes on the AIM code
\begin{itemize}
\item Note that the variable $r_t$ in the code is the deviation of the real rate from the natural rate: $\tilde r_t = i_t -\E_t\pi_{t+1} - r^n_t$. With this note, you can easily map my code back to the exposition in the paper.
\end{itemize}

\section{Result \# 2}
This is the main result of the paper, and it requires solving MNS's heterogeneous agent model. This is a hard problem. Based MNS's online appendix, these are the main steps in solving for equilibrium:

\begin{enumerate}
\item Solve the households' problems using the endogenous grid point method (Carrol, 2006).
\item Simulate the distribution of the households' asset holdings using Young's (2006) nonstochastic histogram method.
\item Checking the equilibrium
\item Updating the initial guess using results from a `simpler' economy.
\end{enumerate}


Useful resources:
\begin{itemize}
\item \url{https://sites.google.com/a/nyu.edu/glviolante/teaching/quantmacro} [which points to some books too]
\end{itemize}

\subsection{Solving the households' problems}
I need to use the endogenous grid point method (EGM) (which is a numerical method for implementing policy function iteration). Note that value function iteration is too slow for this problem. Also note that this method requires approximating the policy function for consumption using a `shape preserving cubic spline'.

\subsubsection{EGM}
\begin{itemize}
\item What I really need to try to understanding is EGM. QuantEcon has one \href{https://lectures.quantecon.org/jl/egm_policy_iter.html}{lecture} on it, but not sure if it's super useful. Here are some other \href{https://www.cemfi.es/~pijoan/Teaching_files/Notes%20on%20endogenous%20grid%20method.pdf}{notes}.
\end{itemize}


\subsubsection{Approximating the policy function}
\begin{itemize}
\item QuantEcon recommends the \href{https://github.com/JuliaMath/Interpolations.jl}{Interpolations} package in Julia. Not sure if it implements shape preserving cubic splines.
\item I think Rudd and MF have notes on approximations using splines.
\end{itemize}



\subsection{Simulating the distribution of asset holdings}

\subsection{Checking equilibrium}

\subsection{Updating the guess}










\end{document}
