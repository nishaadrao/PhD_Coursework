\documentclass[12pt]{article}

%useful packages
\usepackage{color,soul}
\usepackage[usenames,dvipsnames,svgnames,table]{xcolor}
\usepackage{amsmath,amsthm,amscd,amssymb,bm}
\usepackage{hyperref}
\hypersetup{
    colorlinks=true,
    linkcolor=JungleGreen,
    urlcolor  =JungleGreen,
    citecolor = JungleGreen,
    anchorcolor = JungleGreen
}
\usepackage[utf8]{inputenc}
\usepackage[top=2cm, bottom=3cm, left=2cm, right=2cm]{geometry}
\usepackage{pgfplots}
\usepackage{enumitem}
\usepgfplotslibrary{fillbetween}
\usetikzlibrary{patterns}
\usepackage{tcolorbox}
\usepackage{centernot}
\usepackage{mathtools}
\usepackage{xcolor}

%personal definitions and commands
\newcommand{\R}{\mathbb{R}} 
\newcommand{\E}{\mathbb{E}}
\newcommand{\V}{\mathbb{V}}
\newcommand{\C}{\mathbb{C}}
\newcommand{\Prob}{\mathbb{P}}
\newcommand{\e}{\epsilon}
\newcommand\numberthis{\addtocounter{equation}{1}\tag{\theequation}} %allows numbering of single equations in align* environment
\newcommand{\mtx}[1]{\ensuremath{\bm{\mathit{#1}}}}
\newcommand{\B}{\hat{\boldsymbol{\beta}}}
\newcommand{\Cov}{\mathbb{C}\text{ov}}
\newcommand{\N}{\mathcal{N}}



\title{Notes on MNS (2016)}
\author{Anirudh Yadav}
\setlength\parindent{0pt}
\begin{document}

\maketitle

%\setcounter{tocdepth}{2}
%\tableofcontents

\section{Solving the households' problems}
I need to solve the HH problem using the the endogenous grid point method (EGM) (which is a numerical method for implementing policy function iteration) [see my notes below].\\

Each household faces the same problem, so I will suppress the $h$ subscripts. They solve
\begin{align*}
&\max_{c,\ell, b_{t+1}} \sum_t \beta^t \left(\frac{c_t^{1-\gamma}}{1-\gamma} - \frac{\ell_t^{1+\psi}}{1+\psi}\right)\\
&\text{s.t. } c_t + \frac{b_{t+1}}{1+r_t} = b_t + W_tz_t\ell_t - \tau_t\bar\tau(z_t) +D_t\\
&\text{\& } b_{t+1} \geq 0.
\end{align*}

The HH's Euler equation is standard
\begin{align}
c_t^{-\gamma} \geq \beta(1+r_t) \sum_{z_{t+1}}\Pr(z_{t+1}|z_t)c_{t+1}^{-\gamma}  \label{eq:hh1}
\end{align}
And the intratemporal labor supply condition is also standard
\begin{align}
\ell_t^\psi &= W_tz_tc_t^{-1/\gamma} \label{eq:hh2}\\
\implies  \ell_t&=c_t^{-\gamma/\psi}(W_tz_t)^{1/\psi}
\end{align}

\subsection{Initial guess of the household's policy function}
For the EGM method, I need an initial guess of the policy function. Let's keep this as simple as possible: guess that we start in the steady state, and that households consume all of their available resources and save nothing. That is
\begin{align*}
c =  b + \bar Wz\ell - \tau\bar\tau(z) +\bar D
\end{align*}
Substituting the labor supply condition into the above expression, and rearranging a little gives
\begin{align}
c = c^{-\gamma/\psi}(\bar Wz)^{\frac{1+\psi}{\psi}}+ b - \tau\bar\tau(z) +\bar D \label{eq:hh3}
\end{align}
Thus, the initial guess for the policy function, denoted by $g^0(b,z)$, is the level of $c$ that solves (\ref{eq:hh3}). \textcolor{blue}{(It looks like I'll need to use some nonlinear solver to get $c$.).} So I have an initial consumption level for each $(b,z)$ pair.

\subsection{Iterating on the Euler equation}
For starters, assume that the Euler equation holds with equality (I'll deal with the borrowing constraint properly below). Then, with my initial guess of the policy function, I can write the Euler as
\begin{align*}
c^{-1/\gamma} = \beta(1+r) \sum_{z'}\Pr(z'|z)(g_0(b',z'))^{-1/\gamma}.
\end{align*}
Then, today's consumption is simply
\begin{align}
c = \left( \beta(1+r) \sum_{z'}\Pr(z'|z)(g_0(b',z'))^{-1/\gamma}\right)^{-\gamma} \label{eq:hh4}
\end{align}
Now, recall that the HH optimally chooses $b_{t+1}$, with $b_t$ is predetermined. Thus, (\ref{eq:hh4}) gives today's consumption for someone who drew income $z$ today, and optimally chose $b'$. So, we can write (\ref{eq:hh4}) as
\begin{align}
\tilde g^0(b',z) = \left( \beta(1+r) \sum_{z'}\Pr(z'|z)(g_0(b',z'))^{-1/\gamma}\right)^{-\gamma} \label{eq:hh5}
\end{align}
Next, define a grid over values of \textit{tomorrow's} bond holdings $b'$ and call it $B = \{b_1, ...,b_{n_b}\}$, with $b_1 = 0$. Also suppose we have a grid of possible productivity draws, $Z=\{z_1, ..., z_{n_z}\}$, and an associated transition matrix $\Gamma$. Then we can write (\ref{eq:hh5}) as
\begin{align}
\tilde g^0(b_i,z_j) = \left( \beta(1+r) \sum_{\ell}\Gamma_{j\ell} (g_0(b',z'))^{-1/\gamma}\right)^{-\gamma} \label{eq:hh6}, 
\end{align}
which gives the optimal policy $c$, for an agent with current productivity $z_j$, and who optimally chooses tomorrow's bond holdings $b_i$. Thus, for each $(b_i, z_j)$ pair I can get the associated optimal consumption today using (\ref{eq:hh6}).\\

Next, I can use the budget constraint to back out \textit{today's} bond holdings $b_t$:
\begin{align*}
b_t = c_t + \frac{b_{t+1}}{1+r_t} -  W_tz_t\ell_t + \tau_t\bar\tau(z_t) - D_t
\end{align*}
Using the intratemporal labor supply condition (\ref{eq:hh2}) to substitute in for $\ell_t$, gives
\begin{align*}
b_t = c_t + \frac{b_{t+1}}{1+r_t} -  c^{-\gamma/\psi}(\bar Wz)^{\frac{1+\psi}{\psi}} + \tau_t\bar\tau(z_t) - D_t
\end{align*}

Using the grid indicies, and my guess for today's consumption, I can write:
\begin{align*}
b^*_{i,j} = \tilde g^0(b_i,z_j) + \frac{b_i}{1+r} - \left(\tilde g^0(b_i,z_j)\right)^{-\gamma/\psi} (\bar Wz_j)^{\frac{1+\psi}{\psi}} + \tau\bar\tau(z_j) - \bar D ,
\end{align*}
where $b^*_{i,j}$ denotes the bond holdings today for a household who gets a productivity draw $z_j$ today, and who optimally chooses tomorrow's level of bond holdings, $b_j$. Then, we can finally define the `endogenous grid' of today's bond holdings (for each level of productivity $z_j$):
\begin{align*}
B_j^* = \{ b^*_{1,j}, ... , b^*_{n_b,j}\}.
\end{align*}
Next, note that for each grid point in $B_j^*$, I have the updated guess of the optimal policy:
\begin{align*}
g^1(b^*_{i,j},z_j) =  \tilde g^0(b_i,z_j)
\end{align*}
Note that the grid points in $B_j^*$ need not (and will not, in general) coincide with the points in the original grid $B$. Thus, to get the updated guesses $g^1(b_i,z_j)$ for each $b_i \in B$ I need to interpolate $g^1(b^*_{i,j},z_j)$ (see below).\\

Now, I need to deal with the borrowing constraint, $b_{t+1} \geq 0$. Recall that $b^*_{i,j}$ is today's bond holdings for someone with current productivity $z_j$ and who optimally chooses $b_i$. Suppose the borrowing constraint binds;  then agent optimally chooses $b_{t+1} = b_1 =0$. Thus,
\begin{align*}
b^*_{1,j} = \tilde g^0(b_i,z_j) - \left(\tilde g^0(b_i,z_j)\right)^{-\gamma/\psi} (\bar Wz_j)^{\frac{1+\psi}{\psi}} + \tau\bar\tau(z_j) - \bar D.
\end{align*}
Recall that $\tilde g_0(b_1,z_j)$ is our guess of optimal consumption today. Thus, $b^*_{1,j}$ represents the \textit{highest} level of bond holdings \textit{today} for someone with current productivity $z_j$, such that the borrowing constraint binds. Then, for all grid points $b_i \leq b^*_{1,j}$, when $z=z_j$, the borrowing constraint must bind (i.e. $b_{t+1}=0$). Then, we can use the budget constraint to get the updated guess of today's optimal consumption for constrained agents:
\begin{align*}
g^{1,c}(b_i,z_j) = b_i + \left(g^{1,c}(b_i,z_j)\right)^{-\gamma/\psi} (\bar Wz_j)^{\frac{1+\psi}{\psi}} - \tau\bar\tau(z_j) +\bar D \text{ for all } b_i \leq b^*_{1,j},
\end{align*}
where $c$ superscript denotes `constrained'. \textcolor{blue}{Again, I'll need to use a nonlinear solver here.}\\

Putting all this together, the updated guess of the optimal policy function is
\begin{align*}
g^1(b_i,z_j) = 
\begin{cases}
\tilde g^0(b_i,z_j) &\text{ if } b_i > b^*_{1,j} \text{ (and appropriately interpolated)}\\
g^{1,c}(b_i,z_j)&\text{ otherwise.}
\end{cases}
\end{align*}

\subsection{Other details for computation}
\subsubsection{The Markov chain for productivity}
I think I found the Markov chain for productivity in MNS's \verb|initparams.m| file (in the `\verb|parameters|' folder). It looks like the grid is given by
\begin{align*}
Z = (0.493, 1, 2.031)
\end{align*}
with transition matrix
\begin{align*}
\Gamma = 
\begin{bmatrix}
0.966 &	0.0338 &	0.00029 \\
0.017&	0.966& 	0.017 \\
0.0003 &	0.0337 &	0.966
\end{bmatrix}
\end{align*}

\subsubsection{Details I need to iron out}
\begin{itemize}
\item What are the steady state values of $W, D, r, \tau$? Can I solve for these by hand? I need these values for the above procedure.
\item What is the exact form of the function $\bar \tau(z_j)$? Maybe I can leave this out, along with steady state $\tau_t$?
\end{itemize}




\newpage

\section{Overview and results}
Main results:
\begin{enumerate}
\item In standard NKM, the response of (current) output/consumption to forward guidance is too big.
\item In a NKM with idiosyncratic labor income shocks and borrowing constraints, the response of output/consumption to forward guidance is far lower.
\item Something about ZLB [fill in]
\end{enumerate}

\subsection{Intuition [for my own understanding]}

\section{Result \#1}
I've replicated this in AIM. Some notes on the AIM code
\begin{itemize}
\item Note that the variable $r_t$ in the code is the deviation of the real rate from the natural rate: $\tilde r_t = i_t -\E_t\pi_{t+1} - r^n_t$. With this note, you can easily map my code back to the exposition in the paper.
\end{itemize}

\section{Result \# 2}
This is the main result of the paper, and it requires solving MNS's heterogeneous agent model. Note this their model is essentially just a Bewley/Hugget/Aiygari model with sticky prices. This is a hard problem. Based MNS's online appendix, these are the main steps in solving for equilibrium:

\begin{enumerate}
\item Solve the households' problems using the endogenous grid point method (Carrol, 2006).
\item Simulate the distribution of the households' asset holdings using Young's (2006) nonstochastic histogram method.
\item Checking the equilibrium
\item Updating the initial guess using results from a `simpler' economy.
\end{enumerate}



Useful resources:
\begin{itemize}
\item \url{https://sites.google.com/a/nyu.edu/glviolante/teaching/quantmacro} [which points to some books too]
\end{itemize}
The following QuantEcon lectures are probably going to be useful
\begin{itemize}
\item \href{https://lectures.quantecon.org/jl/finite_markov.html}{Markov chains}
\item \href{https://lectures.quantecon.org/jl/egm_policy_iter.html}{EGM}
\item \href{https://lectures.quantecon.org/jl/discrete_dp.html}{Implementing a discrete state dynamic program}
\item \href{https://lectures.quantecon.org/jl/ifp.html}{Hugget model}
\item \href{https://lectures.quantecon.org/jl/aiyagari.html}{Aiyagari model}
\end{itemize}


\subsection{Solving the households' problems}
I need to use the endogenous grid point method (EGM) (which is a numerical method for implementing policy function iteration). Note that value function iteration is too slow for this problem. Also note that this method requires approximating the policy function for consumption using a `shape preserving cubic spline'.\\

\textbf{\textcolor{blue}{A good place to start may be to write down and program the Bewley/Hugget model. Then extend to Aiygari, which has a firm. These models are obviously more basic than MNS, but at least I can get a decent handle on using grids, EGM and deriving the stationary distribution of asset holdings.}}

\subsubsection{EGM}
Here's roughly how EGM works. \textcolor{red}{[I need to understand how exactly this relates to `time iteration'; and the fact that MNS assume that the economy returns to steady state after 250 periods.]} \\

Consider the simplified consumer's problem from MNS (i.e. I've assumed exogenous income rather than a wage and a labor choice)
\begin{align*}
&\max \sum_t \beta^t u(c_t) \\
&\text{s.t. } c_t + \frac{b_{t+1}}{1+r_t} = b_t + z_t\\
&\text{\& } b_{t+1} \geq 0.
\end{align*}
The Euler equation is (you can check handwritten notes for the details):
\begin{align*}
u'(c_t) \geq \beta(1+r_t) \sum_{z_{t+1}}\Pr(z_{t+1}|z_t)u'(c_{t+1})
\end{align*}
For simplicity assume that the solution is interior (i.e. $b_{t+1} > 0$) so that the Euler holds with equality (note that I'll need to figure out how to deal with the constraint too!). I want to implement EGM, which is a policy function iteration method (c.f. value function iteration). Let $g_0(b,z)$ denote our initial guess of the optimal (consumption) policy for a given state pair $(b,z)$. (Note that we are trying to solve for the true policy function $g(b,z)$). Then, I can write the Euler as:
\begin{align*}
u'(c) = \beta(1+r) \sum_{z'}\Pr(z'|z)u'(g_0(b',z')).
\end{align*}
Then, today's consumption is simply
\begin{align}
c = u'^{-1}\left( \beta(1+r) \sum_{z'}\Pr(z'|z)u'(g_0(b',z'))\right) \label{eq:egm1}
\end{align}
Now, recall that the HH optimally chooses $b_{t+1}$, which $b_t$ is predetermined. Thus, (\ref{eq:egm1}) gives today's consumption for someone who drew income $z$ today, and optimally chose $b'$. So, we can write (\ref{eq:egm1}) as
\begin{align}
\tilde g_0(b',z) = u'^{-1}\left( \beta(1+r) \sum_{z'}\Pr(z'|z)u'(g_0(b',z'))\right)  \label{eq:egm2}
\end{align}
Next, define a grid over values of \textit{tomorrow's} bond holdings $b'$ and call it $B = \{b_1, ...,b_{n_b}\}$, with $b_1 = 0$. Also suppose we have a grid of possible income draws, $Z=\{z_1, ..., z_{n_z}\}$, and an associated transition matrix $\Gamma$. Then we can write (\ref{eq:egm2}) as
\begin{align}
\tilde g_0(b_i,z_j) = u'^{-1}\left( \beta(1+r) \sum_{\ell}\Gamma_{j\ell} u'(g_0(b_i, z_\ell))\right) \label{eq:egm3}, 
\end{align}
which gives the optimal policy $c$, for an agent with current income $z_j$, and who optimally chooses tomorrow's bond holdings $b_i$. Thus, for each $(b_i, z_j)$ pair we can get the associated optimal consumption today using (\ref{eq:egm3}).\\

Next, we can use the budget constraint to back out \textit{today's} bond holdings $b_t$:
\begin{align*}
b_t = c_t + \frac{b_{t+1}}{1+r_t} - z_t\\
\end{align*}
Using the grid indicies, we can write:
\begin{align*}
b^*_{i,j} = \tilde g_0(b_i,z_j) + \frac{b_i}{1+r} - z_j,
\end{align*}
where $b^*_{i,j}$ defines bond holdings today for a HH who gets an income draw $z_j$ today, and who optimally chooses tomorrow's level of bond holdings, $b_j$. Then, we can finally define the `endogenous grid' of today's bond holdings (for each level of income $z_j$):
\begin{align*}
B_j^* = \{ b^*_{1,j}, ... , b^*_{n_b,j}\}.
\end{align*}
Next, note that for each grid point in $B_j^*$ we have the updated guess of the optimal policy:
\begin{align*}
g_1(b^*_{i,j},z_j) =  \tilde g_0(b_i,z_j)
\end{align*}
Note that the grid points in $B_j^*$ need not (and will not, in general) coincide with the points in the original grid $B$. Thus, to get the updated guesses $g_1(b_i,z_j)$ for each $b_i \in B$ we need to interpolate $g_1(b^*_{i,j},z_j)$ (see below).\\

Now, let's try to deal with the borrowing constraint, $b_{t+1} \geq 0$. Recall that $b^*_{i,j}$ is today's bond holdings for someone with current income $z_j$ and who optimally chooses $b_i$. Suppose the borrowing constraint binds;  then agent optimally chooses $b_{t+1} = b_1 =0$. Thus,
\begin{align*}
b^*_{1,j} = \tilde g_0(b_1,z_j) - z_j.
\end{align*}
Recall that $\tilde g_0(b_1,z_j)$ is our guess of optimal consumption today. Thus, $b^*_{1,j}$ represents the \textit{highest} level of bond holdings \textit{today} for someone with current income $z_j$, such that the borrowing constraint binds. Then, for all grid points $b_i \leq b^*_{1,j}$, when $z=z_j$, the borrowing constraint must bind (i.e. $b_{t+1}=0$). Then, we can use the budget constraint to get the updated guess of today's optimal consumption for constrained agents:
\begin{align*}
g_1^{constrained}(b_i,z_j) = b_i + z_j \text{ for all } b_i \leq b^*_{1,j}
\end{align*}
Putting all this together, the updated guess of the optimal policy function is
\begin{align*}
g_1(b^*_{i,j},z_j) = 
\begin{cases}
\tilde g_0(b_i,z_j) &\text{ if } b_i > b^*_{1,j}\\
b_i + z_j &\text{ otherwise.}
\end{cases}
\end{align*}

Notes on EGM:
\begin{itemize}
\item \href{https://lectures.quantecon.org/jl/egm_policy_iter.html}{QuantEcon} (the `coleman\_egm' function computes the updated guess of the policy funtion, and the 'check\_convergence' function implements the iteration).
\item \href{https://www.cemfi.es/~pijoan/Teaching_files/Notes%20on%20endogenous%20grid%20method.pdf}{CEMFI} (I summarized these above).
\end{itemize}


\subsubsection{Interpolating the policy function}
\begin{itemize}
\item QuantEcon recommends the \href{https://github.com/JuliaMath/Interpolations.jl}{Interpolations} package in Julia. Not sure if it implements shape preserving cubic splines \textcolor{red}{But maybe I can just start with their in-built linear interpolation?}
\item I think Rudd and MF have notes on approximations using splines.
\end{itemize}

\subsubsection{The Markov Chain for idiosyncratic wage risk}
MNS discretize an AR(1) process for $z$ to a three-point Markov chain. I think it's OK to skip the details of this step given the time constraint. But I still need to get the actual transition matrix they used somehow!
\begin{itemize}
\item MNS use the \verb|rouwen.m| function to get the transition matrix (in the \verb|initparams.m| file) \href{https://sites.google.com/site/dlkhagva/research/matlab-codes-for-the-rouwenhorst-method}{Link}.
\item QuantEcon has an inbuilt \verb|rouwenhorst| function. \href{https://github.com/QuantEcon/QuantEcon.jl/blob/master/src/markov/markov_approx.jl}{Link}.
\item I think $z$ is mean zero.
\end{itemize}

\subsubsection{CH suggestions}
\begin{itemize}
\item Focus on solving the HH's problems before even thinking about the other steps of the solution.
\item For the initial guess of the policy function, suppose that all households consume their idiosyncratic wage income, steady state dividends and bond holdings (i.e. they save nothing). 
\end{itemize}




\subsection{Simulating the distribution of asset holdings}

\subsection{Checking equilibrium}

\subsection{Updating the guess}










\end{document}
