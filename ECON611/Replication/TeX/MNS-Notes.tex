\documentclass[12pt]{article}

%useful packages
\usepackage{color,soul}
\usepackage[usenames,dvipsnames,svgnames,table]{xcolor}
\usepackage{amsmath,amsthm,amscd,amssymb,bm}
\usepackage{hyperref}
\hypersetup{
    colorlinks=true,
    linkcolor=JungleGreen
}
\usepackage[utf8]{inputenc}
\usepackage[top=2cm, bottom=3cm, left=2cm, right=2cm]{geometry}
\usepackage{pgfplots}
\usepackage{enumitem}
\usepgfplotslibrary{fillbetween}
\usetikzlibrary{patterns}
\usepackage{tcolorbox}
\usepackage{centernot}
\usepackage{mathtools}
\usepackage{xcolor}

%personal definitions and commands
\newcommand{\R}{\mathbb{R}} 
\newcommand{\E}{\mathbb{E}}
\newcommand{\V}{\mathbb{V}}
\newcommand{\C}{\mathbb{C}}
\newcommand{\Prob}{\mathbb{P}}
\newcommand{\e}{\epsilon}
\newcommand\numberthis{\addtocounter{equation}{1}\tag{\theequation}} %allows numbering of single equations in align* environment
\newcommand{\mtx}[1]{\ensuremath{\bm{\mathit{#1}}}}
\newcommand{\B}{\hat{\boldsymbol{\beta}}}
\newcommand{\Cov}{\mathbb{C}\text{ov}}
\newcommand{\N}{\mathcal{N}}



\title{Notes on MNS (2016)}
\author{Anirudh Yadav}
\setlength\parindent{0pt}
\begin{document}

\maketitle

%\setcounter{tocdepth}{2}
%\tableofcontents


\section{Overview and results}
Main results:
\begin{enumerate}
\item In standard NKM there is an outsized response of output/consumption to forward guidance.
\item In a NKM with idiosyncratic labor income shocks and borrowing constraints, the response of output/consumption to forward guidance is far lower.
\item Something about ZLB [fill in]
\end{enumerate}

\subsection{Intuition [for my own understanding]}

\section{Result \#1}
I've replicated this in AIM. Some notes on the AIM code
\begin{itemize}
\item Note that the variable $r_t$ in the code is the deviation of the real rate from the natural rate: $\tilde r_t = i_t -\E_t\pi_{t+1} - r^n_t$. With this note, you can easily map my code back to the exposition in the paper.
\end{itemize}

\section{Result \# 2}
This is the main result of the paper, and it requires solving MNS's heterogeneous agent model. This is a hard problem. Based on what I've read so far:
\begin{itemize}
\item Value function iteration is too slow for this kind of problem (rate of convergence is $\beta$).
\item I think I need to use some kind of approximation method (e.g. splines) to approximate the policy function for consumption [I think this is referred to as `projection' (as opposed to perturbation)]
\item I also somehow need to simulate the distribution of asset holdings (e.g. with a histogram?)
\item The weird part is how to update the guessed solution to move it closer to an equilibrium...
\end{itemize}
Useful resources:
\begin{itemize}
\item \url{https://sites.google.com/a/nyu.edu/glviolante/teaching/quantmacro} [which points to some books too]
\end{itemize}










\end{document}
