\documentclass[12pt]{article}

%useful packages
\usepackage{color,soul}
\usepackage[usenames,dvipsnames,svgnames,table]{xcolor}
\usepackage{amsmath,amsthm,amscd,amssymb,bm}
\usepackage{hyperref}
\hypersetup{
    colorlinks=true,
    linkcolor=JungleGreen,
    urlcolor  =JungleGreen,
    citecolor = JungleGreen,
    anchorcolor = JungleGreen
}
\usepackage[utf8]{inputenc}
\usepackage[top=2cm, bottom=3cm, left=2cm, right=2cm]{geometry}
\usepackage{pgfplots}
\usepackage{enumitem}
\usepgfplotslibrary{fillbetween}
\usetikzlibrary{patterns}
\usepackage{tcolorbox}
\usepackage{centernot}
\usepackage{mathtools}
\usepackage{xcolor}

%personal definitions and commands
\newcommand{\R}{\mathbb{R}} 
\newcommand{\E}{\mathbb{E}}
\newcommand{\V}{\mathbb{V}}
\newcommand{\C}{\mathbb{C}}
\newcommand{\Prob}{\mathbb{P}}
\newcommand{\e}{\epsilon}
\newcommand\numberthis{\addtocounter{equation}{1}\tag{\theequation}} %allows numbering of single equations in align* environment
\newcommand{\mtx}[1]{\ensuremath{\bm{\mathit{#1}}}}
\newcommand{\B}{\hat{\boldsymbol{\beta}}}
\newcommand{\Cov}{\mathbb{C}\text{ov}}
\newcommand{\N}{\mathcal{N}}



\title{Replication of `The Power of Forward Guidance Revisited' by McKay, Nakamura \& Steinnson (2016)}
\author{Anirudh Yadav}
\setlength\parindent{0pt}
\begin{document}

\maketitle

\setcounter{tocdepth}{2}
\tableofcontents

\newpage

\section{Overview}
In the basic New Keynesian model (NKM) studied in class the output and inflation response to forward guidance is implausibly large. A potential reason for this `forward guidance puzzle' is the complete markets assumption, which allows the representative agent to take full advantage of any opportunity to intertemporally substitute consumption. McKay, Nakamura and Steinnson's (2016, henceforth MNS) main question is whether the output response to forward guidance is smaller in a NKM with idiosyncratic income risk and incomplete markets. It turns out that it is. The main intuition for their result is that in the model with incomplete markets agents exhibit precautionary savings behaviour. The desire to save for bad times mitigates agents' desire to move consumption forward, dampening the aggregate output response relative to the basic NKM.

\section{Forward guidance in the basic New Keynesian Model}
The first part of the paper shows why forward guidance is so powerful in the basic NKM. Consider the plain vanilla NKM studied in class:
\begin{align*}
y_t &= \E_t[y_{t+1}] - \sigma(i_t - \E_t[\pi_{t+1}] - r_t^n) &\text{ `NK IS curve'}\\
\pi_t  &= \beta\E_t[\pi_{t+1}]  + \kappa y_t &\text{ `NKPC'}
\end{align*}
where the variables have the usual interpretation. Next, suppose the monetary policy rule is given by
\begin{align*}
r_t = i_t - \E_t[\pi_{t+1}] = r_t^n + \e_{t,t-j},
\end{align*}
where $\e_{t,t-j}$ is a monetary shock in period $t$ that is announced in period $t-j$. Suppose the economy starts in steady state and the central bank announces that the real rate will be lower by 1 percentage point for a single quarter, one year in the future; i.e. $\e_{t+4,t} = -0.01.$ To implement this type of shock in AIM, I add the following $MA$ state variables (as discussed in class) to the standard setup:
\begin{align*}
\tilde{r}_t \equiv r_t - r_t^n &= a_1MA_t^1 + a_2MA_t^2 + a_2MA_t^3 + a_4MA_t^4 + a_5MA_t^5
\end{align*}
with
\begin{align*}
MA_t^1 &= \e_{t+4,t}=
\begin{cases}
-0.01 &\text{ if } t=1 \\
0 &\text{ otherwise.}
\end{cases}
\end{align*} 
and
\begin{align*}
MA^j_t = MA_{t-1}^{j-1}
\end{align*}
and $a_1=a_2=a_3=a_4=0$, $a_5 = 1$.\\

Figure 1 plots the response of the output gap to this shock for $\sigma=1$ (other parameter values are standard), replicating Figure 1 in MNS. To understand why the output response is so big, consider the representative agent's linearized Euler equation:
\begin{align*}
\E_t[\Delta\tilde{c}_{t+1}] = \beta \tilde{r}_t 
\end{align*}
which dictates that consumption moves in lock-step with the real rate. Accordingly, with no borrowing constraint, the representative agent can take full advantage of the opportunity to intertemporally substitute.

\begin{figure}[htpb!]
 \centering
 	\caption{Response of Output to Forward Guidance in the Basic NKM}
        \includegraphics[width=0.7\textwidth]{IR_1_R.eps}
\end{figure}



\section{MNS's model}
MNS's model is essentially the basic NKM combined with the main elements of the Huggett model.

\subsection{Households}
There are a continuum of households indexed by $h$. The households' problem is
\begin{align*}
&\max_{\{c,\ell, b_{t+1}\}} \E_0\sum_{t=0}^\infty \beta^t \left(\frac{c_{h,t}^{1-\gamma}}{1-\gamma} - \frac{\ell_{h,t}^{1+\psi}}{1+\psi}\right)\\
\\
&\text{s.t. } c_{h,t} + \frac{b_{h,t+1}}{1+r_t} = b_{h,t} + W_tz_{h,t}\ell_{h,t} - \tau_t\bar\tau(z_{h,t}) +D_t\\
\\
&\text{\& } b_{h,t+1} \geq 0.
\end{align*}
where $z_{h,t}$ is the household's idiosyncratic productivity, which follows a Markov chain with transition probabilities $\Pr(z_{h,t+1}|z_{h,t})$. Households are taxed according to their labor productivity and each receives an equal dividend $D_t$ from intermediate goods firms. Other variables have the standard interpretation.

\subsection{Production}
The production side of the economy is almost identical to the basic NKM studied in class so I suppress the details for brevity. The only differences from the NKM studied in class are: (i) there is no capital; and (ii) intermediate goods firms have linear production technology in labor.

\subsection{Government}
The government runs a balanced budget to maintain a level of debt $B$ each period. The government's budget constraint is
\begin{align*}
\frac{B}{1+r_t} + \sum_z \Gamma^z(z)\tau_t\bar\tau(z) = B
\end{align*}
where $\Gamma^z$ is invariant cross-sectional distribution of productivities. The government is not very important for the model: it simply provides a saving vehicle for households, and taxes the required amount in order to pay interest on its debt.

\subsection{Equilibrium}
I outline the main equilibrium conditions of the model below. The aggregate production function is 
\begin{align*}
S_tY_t = \int n_{j,t}dj \equiv N_t
\end{align*} 
where $j$ indexes the intermediate goods producers, $N_t$ is aggregate labor demand, and $S_t \equiv \int_0^1 \left(\frac{p_{j,t}}{P_t}\right)^{\mu/(1-\mu)}$ where $\mu$ is the elasticity of substitution across intermediate goods as usual. $S_t$ evolves according to equation (10) in MNS.\\

Aggregate labor market clearing requires
\begin{align*}
L_t \equiv \int z\ell_t(z,b) d\Gamma_t(z,b) = N_t
\end{align*}
where $L_t$ is aggregate labor supply, $\ell_t(z,b)$ is the household's decision rule for labor, and $\Gamma(z,b)$ is the distribution of households over idiosyncratic states. \\

Bond market clearing requires
\begin{align*}
B = \int b_t(z,b)d\Gamma_t(z,b)
\end{align*}
where $g_t(z,b)$ is the household's decision rule for savings. \\

The aggregate dividend is given by
\begin{align*}
D_t = Y_t - W_tN_t.
\end{align*}
And finally, goods market clearing requires
\begin{align*}
C_t \equiv \int g_t(z,b)d\Gamma_t(z,b) = Y_t,
\end{align*}
where $g_t(z,b)$ is the household's decision rule for consumption.\\

Equilibrium has the standard definition, which I omit for brevity.

\subsection{Calibration}
MNS calibrate the idiosyncratic productivity to match the persistent component of estimated AR(1) wage process in Floden and Linde (2001). Their basline calibration results in a 3-point Markov chain with transition matrix:
\begin{align*}
\mtx{P} = 
\begin{bmatrix}
0.966 &	0.0338 &	0.00029 \\
0.017&	0.966& 	0.017 \\
0.0003 &	0.0337 &	0.966
\end{bmatrix}
\end{align*} 
and a vector of state values given by $Z = \{0.492,1,2.031\} \equiv \{z_l, z_m, z_h\}$. The supply of government bonds, $B$, is calibrated to 1.4 $\times$ annual GDP; since the time period for the model is one quarter, this calibration implies $B = 1.4 \times 4 \times Y_t$.\footnote{This calibration seems to imply that $B$ is potentially time varying. However, in their setup the government maintains a constant level of assets. Accordingly, I'm not sure if I'm interpreting their calibration correctly.}. MNS also assume that only the most productive households pay tax, so that $\bar \tau(z_h) = 1$ and $\bar \tau(z)=0$ otherwise. Other parameter values are standard, and are summarized in Table 1 in MNS.

\section{Steady state}
Unfortunately, I was unable to get the transitional dynamics associated with forward guidance in MNS's model due to the time constraint. Accordingly, I focus on my solution for the steady state of MNS's model, which is not presented in their paper.

\subsection{Solving the household's problem}
The first ingredient that I need in order to solve for steady state is the household's policy function for consumption, which also pins down their savings and labor supply decisions. To solve for the policy function, I follow MNS and use the so-called `endogenous grid method' (EGM). I leave the details to the Appendix, but the basic idea of EGM is to iterate on the household's Euler equation. I start with an initial guess of the policy function $g^0(z,b)$. Then (assuming the household is not at the borrowing constraint) I can use the Euler equation to get optimal consumption today:
\begin{align*}
c^{-\gamma} = \beta(1+\bar r) \sum_{z'}\Pr(z'|z)(g^0(z',b'))^{-\gamma}.
\end{align*}
Next, I can use the intertemporal budget constraint to back out the level of bond holdings that the household must have had today, in order to finance this policy. It is also easy to identify which agents hit the borrowing constraint, in which case I use their budget constraint to determine consumption, rather than the Euler equation (see Appendix for details). \\

In summary, at every iteration EGM spits out an endogenous grid of bond holdings today, and an associated level of consumption, for each level of productivity. In order to map this policy back onto the original grid of bond holdings, I need to use some interpolation technique. I use a linear interpolation for simplicity, but MNS use a cubic spline.

\subsection{Simulating the distribution of bond holdings}
To simulate the distribution of bond holdings I first get a vector of 1 million idiosyncratic productivity draws, using the Markov chain above. I then use the household's policy function to get the level of bond holdings associated with each value of productivity. That is,
\begin{enumerate}
\item At time $t=0$, I set $z_0$ to $z_m \in Z$.
\item For each subsequent period $t+1$, the new state $z_{t+1}$ is drawn using $\mtx{P}$.
\item I then pick an initial level of bond holdings $b_0 = 0$.
\item For each subsequent period $t+1$, the new level of bond holdings is given by $$b_{t+1} = (1+\bar r)[b_t+(\bar{W}z_t)^{1+1/\psi}g(z_t,b_t)^{-\gamma/\psi}+\bar{D}-g(z_t,b_t)]. $$
\end{enumerate}


\subsection{The iterative procedure}
Before outlining my iterative procedure, I write down some useful steady state conditions. First, linear production technology implies that  $\bar Y = \bar N$. Thus, the steady state dividend is $\bar D = \bar Y (1-\bar W) = \bar C (1-\bar W)$. Next, note that the government's budget constraint implies
\begin{align*}
\sum_z \Gamma^z(z)\bar \tau\bar\tau(z) &= B\left[\frac{\bar r}{1+\bar r}\right]\\
&=1.4 \times 4 \times \bar Y \left[\frac{\bar r}{1+\bar r}\right]\\
&=1.4 \times 4 \times \bar C \left[\frac{\bar r}{1+\bar r}\right].
\end{align*}
Note that the ergodic distribution of the Markov chain is given by $\Gamma^z(z) = [0.25,0.5,0.25]$. Given that only the most productive households pay tax, the above expression implies
\begin{align*}
0.25 \times \bar\tau &= 1.4 \times 4 \times \bar C \left[\frac{\bar r}{1+\bar r}\right]\\
\implies \bar \tau &= 4\times 1.4 \times 4 \times \bar C \left[\frac{\bar r}{1+\bar r}\right].
\end{align*}
Now, note that there are four steady state prices, $\bar r, \bar W, \bar D$ and $\bar \tau$. MNS calibrate $\bar r = 0.005$ to get a 2\% annual rate. Furthermore, the steady state wage is given by $\bar W = 1/\mu$, which is also calibrated. Thus, I only need to solve for the steady state dividend and tax rate. My iterative procedure proceeds as follows:
\begin{enumerate}
\item Guess an initial level of steady state dividends and taxes $X^0 = [\bar D^0, \bar\tau^0]$.
\item Using this guess and other steady state prices, compute the household's policy function.
\item Simulate the distribution of bond holdings, using the policy function.
\item Compute aggregate consumption, $\bar C^0$.
\item Compute the implied level of dividends consistent with goods market clearing, $\tilde{D}^0 = \bar{C}^0\times(1-\bar{W})$.
\item Compute the implied level of taxes consistent with the government budget constraint and goods market clearing: $\tilde{\tau}^0 = 4 \times 1.4\times4\times\bar C^0 \times \bar r/(1+\bar r)$. Let $\tilde{X}^0 = [\tilde{D}^0,\tilde{\tau}^0]$.
\item Update the guess of dividends and taxes: $X^1 = \omega  \tilde{X}^0 + (1-\omega)X^0$. I set $\omega=0.5$.
\item Iterate until convergence: $X^n=X^{n-1}$.
\end{enumerate}
I implement the procedure in \verb|Julia|. It takes 28 iterations and about 1 minute to converge.



\subsection{Some results}

\subsection{Problems with my solution}

\newpage

\section{Appendix}
\subsection{Endogenous grid method}





\end{document}
