\documentclass[12pt]{article}

%useful packages
\usepackage{color,soul}
\usepackage[usenames,dvipsnames,svgnames,table]{xcolor}
\usepackage{amsmath,amsthm,amscd,amssymb,bm}
\usepackage{hyperref}
\hypersetup{
    colorlinks=true,
    linkcolor=JungleGreen
}
\usepackage[utf8]{inputenc}
\usepackage[top=2cm, bottom=3cm, left=2cm, right=2cm]{geometry}
\usepackage{pgfplots}
\usepackage{enumitem}
\usepgfplotslibrary{fillbetween}
\usetikzlibrary{patterns}
\usepackage{tcolorbox}
\usepackage{centernot}
\usepackage{mathtools}
\usepackage{xcolor}

%personal definitions and commands
\newcommand{\R}{\mathbb{R}} 
\newcommand{\E}{\mathbb{E}}
\newcommand{\V}{\mathbb{V}}
\newcommand{\C}{\mathbb{C}}
\newcommand{\Prob}{\mathbb{P}}
\newcommand{\e}{\epsilon}
\newcommand\numberthis{\addtocounter{equation}{1}\tag{\theequation}} %allows numbering of single equations in align* environment
\newcommand{\mtx}[1]{\ensuremath{\bm{\mathit{#1}}}}
\newcommand{\B}{\hat{\boldsymbol{\beta}}}
\newcommand{\Cov}{\mathbb{C}\text{ov}}
\newcommand{\N}{\mathcal{N}}



\title{ECON641 -- Problem Set 1}
\author{Anirudh Yadav}
\setlength\parindent{0pt}
\begin{document}

\maketitle

\setcounter{tocdepth}{1}
\tableofcontents

\newpage

\section{Warmup: factor intensity reversals}
First, I outline the small open economy environment of the $2\times 2$ HO model (for my own purposes). 
\begin{itemize}
\item Two goods, 1 and 2.
\item Two factors, $L$ and $K$; with endogenous factor prices $w$ and $r$, respectively.
\item Production technology is the same in both industries, but they may differ in their relative factor intensities.
\item Exogenously given goods prices, $p_1$ and $p_2$ (i.e. the demand side of the economy is pinned down).
\end{itemize}
Roughly speaking, `no factor intensity reversals' (NFIR) means the following: for any vector of factor prices $(w,r)$, the ordering of relative factor intensities in both industries is always the same. For example, in equilibrium the production of good 1 may be more capital intensive than production of good 2; NFIR implies that at any other vector of factor prices, the production of good 1 must always be more capital intensive compared to good 2. We can show that production technology exhibits NFIR if, given $p_1$ and $p_2$, equilibrium factor prices are uniquely pinned down.\\

\subsection{Cobb Douglas}

Cobb Douglas production clearly satisfies NFIR. To see this, suppose that $F_1(K_1,L_1) = AK_1^\alpha L_1^{1-\alpha}$ and $F_2(K_2,L_2) = AK_2^\beta L_2^{1-\beta}$. The first order conditions for the profit maximization problem for industry 1 are standard:
\begin{align}
p_1\alpha AK_1^{\alpha-1}L_1^{1-\alpha} &= r,\\
p_1(1-\alpha) AK_1^{\alpha}L_1^{-\alpha} &= w.
\end{align}
Dividing (2) by (1) gives
\begin{align}
\frac{1-\alpha}{\alpha} k_1 = \frac{w}{r} \implies k_1 = \frac{\alpha}{1-\alpha}\frac{w}{r}, \text{ where } k_1 = K_1/L_1
\end{align}
Now, the zero profit condition in industry 1 is
\begin{align}
rK_1 + w L_1 &= p_1  AK_1^\alpha L_1^{1-\alpha} \nonumber\\
\implies r k_1 + w &= p_1 A k_1^\alpha
\end{align}
Plugging (3) into (4) and rearranging gives
\begin{align}
p_1 = C_\alpha r^\alpha w^{1-\alpha}
\end{align}
where $C_\alpha = \frac{1}{A(1-\alpha)}\left(\frac{1-\alpha}{\alpha}\right)^\alpha$. An analogous derivation for industry 2 gives
\begin{align}
p_2 = C_\beta r^\beta w^{1-\beta}
\end{align}
where $C_\beta = \frac{1}{A(1-\beta)}\left(\frac{1-\beta}{\beta}\right)^\beta$.
Clearly, given $p_1$ and $p_2$, there is a unique solution to (5) and (6), $(w^*,r^*)$, (unless $\alpha = \beta$).\\

Another (perhaps more intuitive) way to establish NFIR would be to use equation (3) and the equivalent expression for industry 2. These expressions imply that in equilibrium:
\begin{align*}
\frac{k_1}{k_2} = \frac{\alpha (1-\beta)}{\beta(1-\alpha)}.
\end{align*}
That is, the relative factor intensities between the two industries is independent of factor prices.\\


\subsection{CES}

CES production \textit{does not} exhibit NFIR. To see this, suppose $F_i(K_i,L_i) = \left[K_i^{\frac{\sigma_i - 1}{\sigma_i}} + L_i^{\frac{\sigma_i - 1}{\sigma_i}}\right]^{\frac{\sigma_i}{\sigma_i - 1}}$ for $i = 1,2$. The FOCs for industry $i$ are
\begin{align}
p_i\left[K_i^{\frac{\sigma_i - 1}{\sigma_i}} + L_i^{\frac{\sigma_i - 1}{\sigma_i}}\right]^{\frac{1}{\sigma_i - 1}}K_i^{-1/\sigma_i} &= r \label{eq:ces1}\\
p_i\left[K_i^{\frac{\sigma_i - 1}{\sigma_i}} + L_i^{\frac{\sigma_i - 1}{\sigma_i}}\right]^{\frac{1}{\sigma_i - 1}}L_i^{-1/\sigma_i} &= w \label{eq:ces1}
\end{align}
Combining these expressions gives
\begin{align*}
k_i^{-1/\sigma_i} &= \frac{r}{w}\\
\implies k_i &= \left(\frac{r}{w}\right)^{-\sigma_i}.
\end{align*}
Thus, in equilibrium, the relative factor intensities between the two industries is
\begin{align*}
\frac{k_1}{k_2} = \left(\frac{r}{w}\right)^{\sigma_2-\sigma_1},
\end{align*}
which clearly depends on factor prices (unless $\sigma_1 = \sigma_2$).\\

\subsection{Leontief}

Clearly the Leontief production function exhibits NFIR. Suppose both industries have the same production function $F(K,L) = \min\{K,L\}$. Then in equilibrium, both industries must have $k_i = 1$. Then, relative factor intensities do not depend on factor prices. More generally, suppose $F_i(K_i,L_i) = \min\{\alpha_iK,\beta_iL\}$. Then in equilibrium, each industry's capital-labor ratio will be $k_i = \beta_i/\alpha_i$. Again, relative factor intensities are independent of factor prices.

\newpage

\section{$2 \times 2 \times 2$ HO Model}

\newpage

\section{Technology growth in a parameterized version of DFS}

\subsection{}
I follow the derivation in EK (2005). We are given the distribution of efficiencies for producing goods $j$ at Home and Foreign:
\begin{align*}
F_i(z)=\Pr[Z_i(j) \leq z ] = \exp(-T_iz^{-\theta})
\end{align*}

\iffalse
Now, if we pick some random $j \in [0,1]$, the probability distribution of its efficiency, $Z_i(j)$, is obviously same as above. Accordingly, we can order goods such that
\begin{align*}
j &= \exp(-T_iz_i(j)^{-\theta})\\
\implies z_i(j) &= \left(-\frac{T_i}{\ln j}\right)^{-1/\theta}.
\end{align*}
(Essentially, think of the index $j$ as the vertical axis of the CDF above).\\
\fi


Now, we want to derive the DFS-type $A(j)$ curve, which is defined as the ratio of $H$'s efficiency of producing $j$ to $F$'s corresponding efficiency.\\

In the EK setup the efficiencies are realizations of a random variable. Accordingly, we think of $j$ as the \textit{probability} that the $H$'s relative efficiency of producing $j$ is less than some number:
\begin{align*}
j &= \Pr\left[\frac{Z}{Z^*} \leq A\right]\\
&=\Pr\left[Z \leq AZ^*\right]\\
&= \int_0^\infty \exp(-T(Az_*)^{-\theta}) f(z_*)dz_*.
\end{align*}
Now,
\begin{align*}
f(z_*) = \frac{d}{dz} \exp(-T^*z^{-\theta}) = \theta T^*z^{-\theta-1}\exp(-T^*z^{-\theta})
\end{align*}
Substituting into the above integral gives
\begin{align*}
j &= T^*\int_0^\infty \exp(-T(Az_*)^{-\theta}) \times \theta z_*^{-\theta-1}\exp(-T^*z_*^{-\theta})dz_*\\
&=T^*\int_0^\infty \exp(-(TA^{-\theta}+T^*)z_*^{-\theta}) \times\theta z_*^{-\theta-1}dz_*\\
&= \frac{T^*}{(TA^{-\theta}+T^*)} \int_0^\infty \exp(-(TA^{-\theta}+T^*)z_*^{-\theta}) \times -\theta z_*^{-\theta-1}(TA^{-\theta}+T^*)dz_*\\
&=\frac{T^*}{(TA^{-\theta}+T^*)} \int_0^\infty \exp(-(TA^{-\theta}+T^*)z_*^{-\theta}) \times \theta z_*^{-\theta-1}(TA^{-\theta}+T^*)dz_*\\
&=\frac{T^*}{(TA^{-\theta}+T^*)},
\end{align*}
since $ \int_0^\infty \exp(-(TA^{-\theta}+T^*)z_*^{-\theta}) \times \theta z_*^{-\theta-1}(TA^{-\theta}+T^*)dz_* = 1$ (because it is the integral of the Frechet pdf with scale $(T^*A^{-\theta}+T)$). Thus, rearranging to get an expression for $A(j)$ gives
\begin{align}
A(j) = \left[\left(\frac{1-j}{j}\right)\frac{T^*}{T}\right]^{-1/\theta}. \label{eq:EK0}
\end{align}

\newpage

\subsection{}
First note that there are no trade costs so that $d_{ni} = 1$ for all $n,i \in \{F,H\}$.\\

Now, we know that within a country, goods will be purchased from the lowest cost source. Since Home has a comparative advantage at lower values of $j$, we know that Home will produce the range of goods $[0,\bar j]$ where
\begin{align*}
\frac{w}{z(\bar j)} = \frac{w^*}{z^*(\bar j)}.
\end{align*}
The LHS of the above expression is the unit cost of producing $\bar j$ at home, and the RHS is the cost of buying the good from Foreign. Rearranging the above expression gives
\begin{align}
\frac{z(\bar j)}{z^*(\bar j)} &= \frac{w}{w^*} \nonumber\\
\implies A(\bar j) &= \omega, \label{eq:EK1}
\end{align}
where $\omega = w/w^*$. Similarly, Foreign will produce a range of goods $[\underline j, 1]$ domestically, such that
\begin{align}
\frac{w^*}{z^*(\underline j)} &= \frac{w}{z(\underline j)} \nonumber \\
\implies A(\underline j) = \omega. \label{eq:EK2}
\end{align}
Thus, there is a unique cuttoff good.\\

Next, we need to invoke market clearing. Here, we note that preferences are Cobb Douglas, with equal weights across each good. Thus, each country spends a constant share of its income on each good. We know that Home produces $[0,\bar j]$ goods domestically, and exports $[0, \underline j]$ goods to Foreign. Thus, market clearing at Home requires
\begin{align}
wL = \bar j wL + \underline j w^*L^* \label{eq:EK3}\\
\end{align}
Substituting (\ref{eq:EK1}) and (\ref{eq:EK2}) into (\ref{eq:EK3}) and rearranging gives
\begin{align*}
L &= L A^{-1}(\omega) + \frac{1}{\omega} L^* A^{-1}(\omega)\\
\implies \frac{1}{A^{-1}(\omega)} &= 1 + \frac{1}{\omega} \frac{L^*}{L} 
\end{align*}
And, from the above derivation we know: $A^{-1}(\omega) = \frac{T^*}{(T\omega^{-\theta}+T^*)}$. Substituting this into the above expression gives
\begin{align*}
\frac{T\omega^{-\theta}+T^*}{T^*} &= 1 +\frac{1}{\omega} \frac{L^*}{L}\\
\implies \frac{T}{T^*} \omega^{-\theta} &=\frac{1}{\omega} \frac{L^*}{L}\\
\implies \omega &= \left[\frac{T^*}{T}\frac{L^*}{L}\right]^{1/(1-\theta)}.
\end{align*}
And the equilibrium cutoff good is
\begin{align*}
\bar j &= A^{-1}(\omega)
\end{align*}

\subsection{}
With no trade costs goods prices are identical in $H$ and $F$. Thus, $\omega$ measures $H$'s welfare relative to $F$'s. From the above expression for equilibrium $\omega$, if $\theta > 1$, then an increase in $T^*$ reduces relative welfare in $H$. The comparative static is
\begin{align*}
\frac{d\omega}{dT^*} = \left[\frac{1}{T}\frac{L^*}{L}\right]^{1/(1-\theta)} \frac{1}{1-\theta} (T^*)^{\theta/(1-\theta)}.
\end{align*}
So, if $\theta > 1$, then the effect of an increase in $T^*$ is decreasing in $L^*/L$, which is observable in the data.

\newpage

\section{Key implications of EK's Ricardian model}

\subsection{}
The unit cost of sending good $j$ from $i$ to $n$ is given by
\begin{align}
C_{ni}(j) = \frac{c_i}{Z_i(j)}d_{ni} \label{eq:EK21}
\end{align}

\subsection{}
We want to compute the probability that $i$ will sell good $j$ to $n$. Since the derivation below is the same for all goods, I suppress the index $j$. \\

\textbf{(a)} From (\ref{eq:EK21}) note that 
\begin{align*}
Z_i = \frac{c_id_{ni}}{C_{ni}}
\end{align*}
Now,
\begin{align*}
\Pr[Z_i \leq p] &= \Pr[{c_id_{ni}}/{C_{ni}} \leq p]
\end{align*}
Thus,
\begin{align*}
\Pr[Z_i \leq {c_id_{ni}}/{p} ] &=  \Pr[{c_id_{ni}}/{C_{ni}} \leq {c_id_{ni}}/{p}]\\
&= \Pr[p \leq C_{ni}]\\
&= \Pr[C_{ni} \geq p]\\
&= 1- \Pr[C_{ni} \leq p]\\
\therefore \Pr[C_{ni} \leq p] &= 1-\Pr[Z_i \leq {c_id_{ni}}/{p} ]\\
&= 1- F_i(c_id_{ni}/{p})\\
&= 1- \exp(-T_i(c_id_{ni})^{-\theta}p^\theta),
\end{align*}
which is the probability distribution of $C_{ni}$. Denote this distribution as $G_{ni}$ so that
\begin{align}
G_{ni}(p) = 1- \exp(-T_i(c_id_{ni})^{-\theta}p^\theta) \label{eq:EK22}
\end{align}

\textbf{(b)} Next we want to compute the probability that $i$ is the cheapest supplier for $n$. Denote this probability as $\pi_{ni}$. We have
\begin{align}
\pi_{ni} \equiv \Pr[ C_{ni} &= \min\{C_{ns}; s\neq i\}] \nonumber \\
&= \Pr[C_{ns} \geq C_{ni} \text{ for all } s \neq i]\nonumber \\ 
&= \int_0^\infty \prod_{s\neq i}[1-G_{ns}(C_{ni})] dG_{ni}(C_{ni})\nonumber \\
&=  \int_0^\infty \prod_{s\neq i}[1-G_{ns}(p)] dG_{ni}(p), \label{eq:EK23}
\end{align}
where I just re-denote the integration dummy as $p$ to ease notation.\\

\textbf{(c)} Now,
\begin{align}
dG_{ni}(p) = \frac{d}{dp} G_{ni}(p) &= \exp(-T_i(c_id_{ni})^{-\theta}p^\theta) T_i(c_id_{ni})^{-\theta} \theta p^{\theta-1} \label{eq:EK24}
\end{align}
Substituting (\ref{eq:EK22}) and (\ref{eq:EK24}) into (\ref{eq:EK23}) gives
\begin{align*}
\pi_{ni} &= \int_0^\infty\left[ \prod_{s\neq i} \exp(-T_s(c_sd_{ns})^{-\theta}p^\theta) \right] \exp(-T_i(c_id_{ni})^{-\theta}p^\theta) T_i(c_id_{ni})^{-\theta} \theta p^{\theta-1}dp\\
&=  \int_0^\infty\exp(-\sum_{s \neq i} T_s(c_sd_{ns})^{-\theta}p^\theta)  \exp(-T_i(c_id_{ni})^{-\theta}p^\theta) T_i(c_id_{ni})^{-\theta} \theta p^{\theta-1}dp\\
&= \int_0^\infty\exp(-\sum_{i=1}^N T_i(c_{ni}d_{ni})^{-\theta}p^\theta)T_i(c_id_{ni})^{-\theta} \theta p^{\theta-1}dp\\
&=\int_0^\infty\exp(-\Phi_np^\theta) T_i(c_id_{ni}) \theta p^{\theta-1}dp\\
&=T_i(c_id_{ni})^{-\theta} \int_0^\infty\exp(-\Phi_np^\theta) \theta p^{\theta-1}dp\\
&= \frac{T_i(c_id_{ni})^{-\theta} }{\Phi_n} \int_0^\infty\exp(-\Phi_np^\theta) \Phi_n\theta p^{\theta-1}dp\\
&=\frac{T_i(c_id_{ni})^{-\theta} }{\Phi_n},
\end{align*}
since the integral in the second last line evaluates to 1, because it is a pdf of a probability distribution of the form (\ref{eq:EK22}).

\subsection{} Next we want to compute the probability distribution of goods prices actually bought in market $n$. Call this price $P_n$ and recall that $n$ buys the lowest cost good
\begin{align*}
P_n = \min_{i=1,..,N}\{C_{ni}\}
\end{align*}
Thus
\begin{align*}
\Pr[P_n \leq p ] &= \Pr[ \min_{i=1,..,N}\{C_{ni}\} \leq p]\\
&= 1-\Pr[C_{ni} > p \text{ for all } i]\\
&= 1- \prod_{i=1}^N[1-G_{ni}(p)]\\
&= 1-  \prod_{i=1}^N \exp(-T_i(c_id_{ni})^{-\theta}p^\theta) \text{ using (\ref{eq:EK22}),}\\
&= 1- \exp(-\Phi_n p^\theta).
\end{align*}
Denote this distribution as $G_n(p)$, so that 
\begin{align}
G_n(p) \equiv \Pr[P_n \leq p ] = 1- \exp(-\Phi_n p^\theta) \label{eq:EK25}
\end{align}

\subsection{} Next, we want to compute the probability distribution of goods prices that $n$ \textit{actually buys} from country $i$. That is, we want to compute the conditional probability distribution:
\begin{align*}
\Pr[P_n \leq p | P_n = C_{ni}]
\end{align*}
Now,
\begin{align*}
Pr[P_n \leq p | P_n = C_{ni}]&=\Pr[P_n \leq p |C_{ni} = \min\{C_{ns}; s\neq i\}]\\
&=\Pr[C_{ni}  \leq p |C_{ni} = \min\{C_{ns}; s\neq i\}]\\
&=\frac{1}{\pi_{ni}} \int_0^p \prod_{s \neq i}  [1-G_{ns}(q)] dG_{ni}(q)\\
&=\frac{1}{\pi_{ni}} \pi_{ni}\int_0^p \exp(-\Phi_nq^\theta) \Phi_n\theta q^{\theta-1}dq,
\end{align*}
using our previous derivations. Thus,
\begin{align*}
\Pr[P_n \leq p | P_n = C_{ni}]&= \int_0^p \exp(-\Phi_nq^\theta) \Phi_n\theta q^{\theta-1}dq \\
&= \int_0^p dG_n(q)\\
&= G_n(p).
\end{align*}
So, for goods that are purchased in $n$, conditioning on the source does not affect the distribution of the good's price. This result seems at odds with reality, as we discussed in class. One would think that German cars bought in the US would have a different price distribution compared to Japanese cars bought in the US.


\subsection{}
I'm not exactly sure how to do this problem, but here's my best shot.\\

\textbf{(a)} Given CES preferences over the unit mass of goods, $n$'s demand for good $j$ is given by
\begin{align*}
X_n(j) &= \left(\frac{P_n(j)}{P_n}\right)^{1-\sigma}X_n\\
&=  \left(\frac{\min_{i=1,..,N}\{C_{ni}(j)\}}{P_n}\right)^{1-\sigma}X_n
\end{align*}

\textbf{(b)} Then, $n$'s expected expenditure on good $j$, sourced from $i$ is
\begin{align*}
X_{ni}(j) &= \E[X_n(j)  | i*(j) = i] \Pr[i^*(j) = i]\\
&=\E \left[\left(\frac{\min_{i=1,..,N}\{C_{ni}(j)\}}{P_n}\right)^{1-\sigma}X_n|  i^*(j) = i\right] \pi_{ni}\\
&= \E \left[\left(\frac{C_{ni}(j)}{P_n}\right)^{1-\sigma}X_n|  i^*(j) = i\right] \pi_{ni}\\
&= \E \left[\left(\frac{P_n}{P_n}\right)^{1-\sigma}X_n|  i^*(j) = i\right] \pi_{ni}\\
&= \pi_{ni} X_n.
\end{align*}




\subsection{} We've shown that $\pi_{ni}$ is the probability that country $n$ purchases good any good $j$ from $i$. Since there is a unit measure of goods, it follows that $\pi_{ni}$ is the total fraction of the $j \in [0,1]$ goods that are sourced from from $i$ in country $n$. Then, we can split the the total unit measure $j\in [0,1]$ of goods purchased in $n$ into the share supplied by each source country. Recall that conditioning on the source of a good does not affect the distribution of the good's price in $n$. Accordingly, the fraction of $n$'s total expenditure that goes to country $i$ is the same as the fraction of goods that $n$ purchases from country $i$; namely, $\pi_{ni}$.


\section{Quantitative analysis in the EK model}

\subsection{A peek at the data}
See attached \verb|R| code for the aggregation of the WIO data.\\

Table 1 (overleaf) shows the ratio of imports of intermediate goods to total imports for each country in the WIOD. For each country in the sample, imports of intermediate goods accounts for more than half of total imports; the mean intermediate import share is 0.64. These results imply that it is extremely important to incorporate and intermediate goods sector when modelling aggregate trade flows.\\

Table 2 presents the ratio of each country's trade deficit (imports minus exports) to total expenditure. For most countries, the trade deficit is tiny compared to total expenditure; the mean trade deficit to expenditure ratio is 0.002. These results suggest that the usual balanced trade assumption is a reasonable one.

\begin{table}[!htbp]
\caption{\textbf{Intermediate Goods Share of Total Imports by Country}}
\centering
\begin{tabular}{rlr}
  \hline
 & Country & Intermediate trade share \\ 
  \hline
1 & AUS & 0.55 \\ 
  2 & AUT & 0.60 \\ 
  3 & BEL & 0.67 \\ 
  4 & BGR & 0.67 \\ 
  5 & BRA & 0.68 \\ 
  6 & CAN & 0.60 \\ 
  7 & CHN & 0.75 \\ 
  8 & CYP & 0.46 \\ 
  9 & CZE & 0.69 \\ 
  10 & DEU & 0.57 \\ 
  11 & DNK & 0.61 \\ 
  12 & ESP & 0.62 \\ 
  13 & EST & 0.63 \\ 
  14 & FIN & 0.71 \\ 
  15 & FRA & 0.61 \\ 
  16 & GBR & 0.56 \\ 
  17 & GRC & 0.59 \\ 
  18 & HUN & 0.76 \\ 
  19 & IDN & 0.78 \\ 
  20 & IND & 0.81 \\ 
  21 & IRL & 0.76 \\ 
  22 & ITA & 0.63 \\ 
  23 & JPN & 0.61 \\ 
  24 & KOR & 0.76 \\ 
  25 & LTU & 0.58 \\ 
  26 & LUX & 0.78 \\ 
  27 & LVA & 0.48 \\ 
  28 & MEX & 0.70 \\ 
  29 & MLT & 0.64 \\ 
  30 & NLD & 0.68 \\ 
  31 & POL & 0.64 \\ 
  32 & PRT & 0.56 \\ 
  33 & ROM & 0.65 \\ 
  34 & RUS & 0.53 \\ 
  35 & SVK & 0.67 \\ 
  36 & SVN & 0.58 \\ 
  37 & SWE & 0.67 \\ 
  38 & TUR & 0.64 \\ 
  39 & TWN & 0.66 \\ 
  40 & USA & 0.53 \\ 
  41 & RoW & 0.70 \\ 
   \hline
\end{tabular}
\end{table}

\begin{table}[!htbp]
\caption{\textbf{Ratio of Trade Deficit to Total Expenditure by Country}}
\centering
\begin{tabular}{rlrrrrr}
  \hline
 & Country & Total Imports & Total Exports & Total Expenditure & Deficit Ratio* & Trade deficit \\ 
  & & (\$ b) & (\$ b) & (\$ b) &  & (\$ b) \\ 
  \hline
1 & AUS & 82183 & 89119 & 773044 & -0.009 & -6936 \\ 
  2 & AUT & 74129 & 76573 & 326787 & -0.007 & -2444 \\ 
  3 & BEL & 145488 & 159808 & 476141 & -0.030 & -14320 \\ 
  4 & BGR & 5687 & 5496 & 25011 & 0.008 & 191 \\ 
  5 & BRA & 70230 & 63215 & 1102372 & 0.006 & 7015 \\ 
  6 & CAN & 262124 & 318038 & 1274587 & -0.044 & -55914 \\ 
  7 & CHN & 233466 & 278005 & 3079150 & -0.014 & -44539 \\ 
  8 & CYP & 3720 & 1846 & 15698 & 0.119 & 1874 \\ 
  9 & CZE & 34011 & 31977 & 140246 & 0.015 & 2034 \\ 
  10 & DEU & 546757 & 612622 & 3331636 & -0.020 & -65865 \\ 
  11 & DNK & 54834 & 66476 & 256046 & -0.045 & -11642 \\ 
  12 & ESP & 174061 & 139171 & 1136802 & 0.031 & 34890 \\ 
  13 & EST & 3395 & 3174 & 12286 & 0.018 & 221 \\ 
  14 & FIN & 37210 & 50286 & 221058 & -0.059 & -13076 \\ 
  15 & FRA & 327210 & 348182 & 2392238 & -0.009 & -20972 \\ 
  16 & GBR & 371112 & 376887 & 2679956 & -0.002 & -5775 \\ 
  17 & GRC & 39250 & 16839 & 223624 & 0.100 & 22411 \\ 
  18 & HUN & 31956 & 28746 & 106129 & 0.030 & 3210 \\ 
  19 & IDN & 46340 & 64460 & 322243 & -0.056 & -18120 \\ 
  20 & IND & 65785 & 66604 & 900499 & -0.001 & -819 \\ 
  21 & IRL & 71357 & 87495 & 188948 & -0.085 & -16138 \\ 
  22 & ITA & 263034 & 269745 & 2128874 & -0.003 & -6711 \\ 
  23 & JPN & 392472 & 511342 & 8569322 & -0.014 & -118870 \\ 
  24 & KOR & 171804 & 197894 & 1127198 & -0.023 & -26090 \\ 
  25 & LTU & 4512 & 3723 & 19312 & 0.041 & 789 \\ 
  26 & LUX & 22195 & 25249 & 44725 & -0.068 & -3054 \\ 
  27 & LVA & 2702 & 2540 & 14573 & 0.011 & 162 \\ 
  28 & MEX & 172069 & 169818 & 1088325 & 0.002 & 2251 \\ 
  29 & MLT & 3216 & 2494 & 7928 & 0.091 & 722 \\ 
  30 & NLD & 181619 & 212380 & 710528 & -0.043 & -30761 \\ 
  31 & POL & 53523 & 44796 & 341961 & 0.026 & 8727 \\ 
  32 & PRT & 42004 & 28517 & 225600 & 0.060 & 13487 \\ 
  33 & ROM & 12402 & 10830 & 73977 & 0.021 & 1572 \\ 
  34 & RUS & 48576 & 97684 & 396030 & -0.124 & -49108 \\ 
  35 & SVK & 12680 & 12514 & 48242 & 0.003 & 166 \\ 
  36 & SVN & 9557 & 8933 & 39114 & 0.016 & 624 \\ 
  37 & SWE & 88073 & 108985 & 443365 & -0.047 & -20912 \\ 
  38 & TUR & 57821 & 41445 & 518382 & 0.032 & 16376 \\ 
  39 & TWN & 154033 & 170777 & 650183 & -0.026 & -16744 \\ 
  40 & USA & 1314500 & 981035 & 18636780 & 0.018 & 333465 \\ 
  41 & RoW & 1178338 & 1079715 & 6694889 & 0.015 & 98623 \\ 
   \hline
\multicolumn{7}{ p{18cm} }{\footnotesize \textit{Note}$^*$: the deficit ratio is trade deficit divided by total expenditure. }\\
\end{tabular}
\end{table}

\newpage

\section{Gravity} 


\subsection{} The gravity specification from AvW's (2003) Armington model (with symmetric trade costs) is given by
\begin{align}
X_{ni} = \frac{Y_nY_i}{Y_w}\left(\frac{d_{ni}}{P_n P_i}\right)^{1-\sigma}  \label{eq:grav1}
\end{align}
Taking logs
\begin{align*}
\log X_{ni} &= - \log Y_w + \log Y_n + \log Y_i + (1-\sigma)\log d_{ni} - (1-\sigma)P_n - (1-\sigma)P_i 
\end{align*}
Given the parametric form for trade costs, the estimating equation becomes
\begin{align}
\log \left(\frac{X_{ni}}{Y_n Y_i}\right) =& \alpha_0 + \alpha_1 (1-\sigma) \text{log distance}_{ni} +  \alpha_2 (1-\sigma) \text{no contiguity}_{ni} \nonumber\\
&+ \alpha_3 (1-\sigma) \text{no common language}_{ni} + \alpha_4 (1-\sigma) \text{no colonial ties}_{ni} \nonumber\\
&- (1-\sigma)m_n - (1-\sigma)m_i + \nu_{ni}, \label{eq:grav3}
\end{align}
where $m_n$ and $m_i$ are importer and exporter fixed effects, respectively. Note that AvW impose unit income elasticities by using $\log \left(\frac{X_{ni}}{Y_n Y_i}\right)$ as the dependent variable. Santos-Silva and Tenreyro (SST) (2006) note that this is probably not the best approach (see fn 29), but I'll stick with it anyway. For simplicity, I estimate (\ref{eq:grav3}) only for the subsample of observations where $X_{ni} >0$ (I discuss the implications of this below). Table 1 (column (1)) presents the OLS estimates of (\ref{eq:grav3}).

\begin{table}[!htbp]
\centering
\caption{\textbf{Estimated Gravity Parameters Using Different Estimators}}
\begin{tabular}{l*{3}{c}}
\hline
Estimator                    &OLS& PPML &PPML\\
Dependent variable                    & $\ln(X_{ni}/Y_nY_i)$ &$X_{ni}/Y_nY_i>0$&$X_{ni}/Y_nY_i$\\
\hline
log distance              &      -1.607&      -1.115&      -1.409\\
                    &    (0.0235)&    (0.0448)&    (0.0412)\\

no contigity            &      -0.918&      -0.596&      -0.398\\
                    &     (0.103)&     (0.110)&     (0.111)\\

no common language             &      -0.733&      -0.805&      -0.810\\
                    &    (0.0515)&     (0.103)&     (0.116)\\

no colonial ties               &      -1.055&      -0.894&      -0.854\\
                    &    (0.0960)&     (0.160)&     (0.169)\\
Fixed effects & Yes & Yes & Yes\\
\hline
Observations        &       19997&       19997&       26967\\
\hline
\multicolumn{4}{l}{\textit{Notes:} Eiker-White standard errors in parentheses}\\
\end{tabular}
\end{table}


\subsection{}
\textbf{(a)} Clearly a zero trade flow for a country-pair observation poses a problem for the estimation of (\ref{eq:grav3}) because the dependent variable would be equal to $- \infty$. As SST (2006) note, most studies simply drop the zero observations when estimating gravity equations of the form (\ref{eq:grav3}), but Head and Meyer (HM) show that this results in selection bias (because zeros account for a large share of total country-pair observations). Another approach is to estimate the model using $X_{ni}+1$ instead of $X_{ni}$, but SST and HM note that this procedure is bad: it yields inconsistent estimators of the gravity parameters and the results depend on the units of measurement of $X_{ni}$. Another approach is to assume that the observed $X_{ni}$ is truncated at its lowest observed value and estimate (\ref{eq:grav3}) using Tobit regression. SST's proposed PPML method (discussed below) allows for zeros.\\

\textbf{(b)} As SST (2006) mention, for the estimated parameters in (\ref{eq:grav3}) to be unbiased, $\nu_{ni}$ must be statistically independent of the regressors.  \\

\textbf{(c)} The estimated coefficient on log distance is $\widehat{\alpha_1(1-\sigma)}$. That is, we cannot separately identify $\alpha_1$ and $(1-\sigma)$.\\

\textbf{(d)} Comparing (\ref{eq:grav3}) and the equation just above it we can see that the fixed effects parameters are given by:
\begin{align*}
m_i &= (1-\sigma)P_i\\
m_n &= (1-\sigma)P_n
\end{align*}
$P_i$ and $P_n$ are the ideal price indicies in country $i$ and $n$, respectively. With $\sigma>1$ the Armington model predicts that bilateral trade flows should be increasing in $P_i$ and $P_n$. The idea is simple: if $P_n$ is high, then goods are relatively expensive in the destination country, which makes it easier for $i$'s imports to penetrate the market. Similarly, if $P_i$ is high, then it's hard for $i$ to export its good to all destinations, so it's relatively easier for $i$ to sell in $n$ (I'm not sure about this).\\

\textbf{(e)} Table 3 also presents the PPML estimates of the gravity equation, following SST  (column (2) presents results for the positive trade sample and (3) presents results for the full sample). Consistent with SST, the PPML estimates of the coefficient on log distance are lower than the corresponding OLS estimate.



\end{document}
