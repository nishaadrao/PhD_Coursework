\documentclass[12pt]{article}

%useful packages
\usepackage{color,soul}
\usepackage[usenames,dvipsnames,svgnames,table]{xcolor}
\usepackage{amsmath,amsthm,amscd,amssymb,bm}
\usepackage{hyperref}
\hypersetup{
    colorlinks=true,
    linkcolor=JungleGreen,
    urlcolor  =JungleGreen,
    citecolor = JungleGreen,
    anchorcolor = JungleGreen
}
\usepackage[utf8]{inputenc}
\usepackage[top=2cm, bottom=3cm, left=2cm, right=2cm]{geometry}
\usepackage{pgfplots}
\usepackage{enumitem}
\usepgfplotslibrary{fillbetween}
\usetikzlibrary{patterns}
\usepackage{tcolorbox}
\usepackage{centernot}
\usepackage{mathtools}
\usepackage{xcolor}

%personal definitions and commands
\newcommand{\R}{\mathbb{R}} 
\newcommand{\E}{\mathbb{E}}
\newcommand{\V}{\mathbb{V}}
\newcommand{\C}{\mathbb{C}}
\newcommand{\Prob}{\mathbb{P}}
\newcommand{\e}{\epsilon}
\newcommand\numberthis{\addtocounter{equation}{1}\tag{\theequation}} %allows numbering of single equations in align* environment
\newcommand{\mtx}[1]{\ensuremath{\bm{\mathit{#1}}}}
\newcommand{\B}{\hat{\boldsymbol{\beta}}}
\newcommand{\Cov}{\mathbb{C}\text{ov}}
\newcommand{\N}{\mathcal{N}}



\title{Notes on Chatterjee (2018)}
\author{Anirudh Yadav}
\setlength\parindent{0pt}
\begin{document}

\maketitle

\section{Overview}
\begin{itemize}
\item Indian farmers are very poor (median annual income $\approx \$365$).
\item Low farmer revenue is partly due to the low prices they receive for their produce.
\item One potential reason for low prices may be the monopsony power of intermediaries, who are the main buyers of farmers' output in India.
\item A source of intermediaries' market power is policy: farmers are only allowed to sell output to government-licensed intermediaries at regulated market locations in their home state.
\item Chatterjee asks: 
\begin{itemize}
\item does more spatial competition between intermediaries increase the price farmers receive?
\item how does removing the interstate trade restriction affect farmer prices/production/income?
\end{itemize}
\end{itemize}

\section{Data}
Data on:
\begin{itemize}
\item Market locations
\item Crop prices at each market
\item Local controls: rainfall, production, WHAT ELSE?
\item Other stuff...
\end{itemize}


\section{Reduced-form stuff}

\subsection{Empirical methodology}
For the causal estimates, the basic idea is to choose market pairs close to each other but separated by a border. If the markets are close enough, it's like we're comparing apples with apples: demand, soil quality, rainfall, etc. should all be very similar for both markets. Then the only thing driving the difference in prices should be the extent of spatial competition faced by each market.

\subsection{Empirical results}

\newpage

\section{Model}
\subsection{Environment}
\begin{itemize}
\item $S$ regions in the economy
\item One crop (we'll add more later)
\item Within each region,
\begin{itemize}
\item farmers, $f\in \{1,...,F\} \equiv \mathcal{F}$
\item intermediares, $m \in  \{1,...,M\} \equiv \mathcal{M}$
\end{itemize}
\item Iceberg trade costs: $\tau_{fm}>1$.
\end{itemize}

\subsection{Production}
\begin{itemize}
\item Farmers have Cobb-Douglas production technology,
\begin{align*}
y_f = \tilde{A}_f\left(h_f^\gamma l_f^\nu \prod_{k=1}^K(x_f^k)^{\alpha_k}\right),
\end{align*}
\begin{itemize}
\item $\{x^k\}$ are intermediate inputs, with prices $\{w^k\}$ (exogenously given)
\item $h_f$ and $l_f$ are endowments of land and labor (i.e. fixed).
\end{itemize}
\end{itemize}

\subsection{Market choice}
\begin{itemize}
\item Farmer's problem is to choose the market $m$ that maximizes profit:
\begin{align*}
\max_{m \in \mathcal{M}} \left\{\frac{p^f(m)y_f}{\tau_{fm}}\right\}
\end{align*}
\end{itemize}

\subsection{Price determination}
\begin{itemize}
\item Once farmer $f$ reaches market $m$, all costs are sunk.
\item Farmer's price is determined via Nash Bargaining.
\item Farmer's outside option:
\begin{align*}
\underline{p}(m) = \max_{k \in \mathcal{M} \backslash \{m\}} \left\{\frac{p^f(k)}{\tau_{mk}}\right\}
\end{align*}
\end{itemize}





\section{Counterfactuals}







\end{document}
