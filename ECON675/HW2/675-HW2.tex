\documentclass[12pt]{article}

%useful packages
\usepackage{color,soul}
\usepackage[usenames,dvipsnames,svgnames,table]{xcolor}
\usepackage{amsmath,amsthm,amscd,amssymb,bm}
\usepackage{hyperref}
\hypersetup{
    colorlinks=true,
    linkcolor=JungleGreen
}
\usepackage[utf8]{inputenc}
\usepackage[top=2cm, bottom=3cm, left=2cm, right=2cm]{geometry}
\usepackage{pgfplots}
\usepackage{enumitem}
\usepgfplotslibrary{fillbetween}
\usetikzlibrary{patterns}
\usepackage{tcolorbox}
\usepackage{centernot}
\usepackage{mathtools}
\usepackage{xcolor}

%personal definitions and commands
\newcommand{\R}{\mathbb{R}} 
\newcommand{\E}{\mathbb{E}}
\newcommand{\V}{\mathbb{V}}
\newcommand{\C}{\mathbb{C}}
\newcommand{\Prob}{\mathbb{P}}
\newcommand{\e}{\epsilon}
\newcommand\numberthis{\addtocounter{equation}{1}\tag{\theequation}} %allows numbering of single equations in align* environment
\newcommand{\mtx}[1]{\ensuremath{\bm{\mathit{#1}}}}
\newcommand{\B}{\hat{\boldsymbol{\beta}}}
\newcommand{\Cov}{\mathbb{C}\text{ov}}
\newcommand{\N}{\mathcal{N}}



\title{ECON675: Assignment 2}
\author{Anirudh Yadav}
\setlength\parindent{0pt}
\begin{document}

\maketitle

%\setcounter{tocdepth}{2}
%\tableofcontents

\section{Question 1: Kernel Density Estimation}

\subsection{Density derivatives}
I follow the derivation in Hansen's notes. We are interested in estimating
\begin{align*}
f^{(s)}(x) = \frac{d^s}{dx^s}f(x).
\end{align*}
The natural estimator is
\begin{align*}
\hat f^{(s)}(x)= \frac{d^s}{dx^s}\hat f(x)
\end{align*}
Now, we know that $\hat f(x) = \frac{1}{nh}\sum_i K\left(\frac{X_i - x}{h}\right)$. Thus,
\begin{align*}
\hat f^{(1)}(x) &= \frac{-1}{nh^2}\sum_{i=1}^n K^{(1)}\left(\frac{X_i - x}{h}\right),\\
\hat f^{(2)}(x) &= \frac{1}{nh^3}\sum_{i=1}^n K^{(2)}\left(\frac{X_i - x}{h}\right),\\
&\vdots \\
\hat f^{(s)}(x) &= \frac{(-1)^s}{nh^{1+s}}\sum_{i=1}^n K^{(s)}\left(\frac{X_i - x}{h}\right).
\end{align*}
Now,
\begin{align*}
\E[\hat f^{(s)}(x)] &= \frac{1}{n} \sum_{i=1}^n \E\left[ \frac{(-1)^s}{h^{1+s}}K^{(s)}\left(\frac{X_i - x}{h}\right)\right]\\
&=\E\left[ \frac{(-1)^s}{h^{1+s}}K^{(s)}\left(\frac{X_i - x}{h}\right)\right], \text{ since $X_i$ are iid}.\\
&=\int_{-\infty}^{\infty} \frac{(-1)^s}{h^{1+s}}K^{(s)}\left(\frac{z - x}{h}\right)f(z)dz
\end{align*}
Next, we want to use integration by parts: $\int u dv = uv - \int vdu$. Define
\begin{align*}
dv = \frac{(-1)^s}{h^{s}}\frac{1}{h}K^{(s)}\left(\frac{z - x}{h}\right) \implies v = \frac{(-1)^s}{h^{s}}K^{(s-1)}\left(\frac{z - x}{h}\right)
\end{align*}
And
\begin{align*}
u = f(z) \implies du = f^{(1)}(z).
\end{align*}
Thus,
\begin{align*}
\E[\hat f^{(s)}(x)] &= \left[ \frac{(-1)^s}{h^{s}}K^{(s-1)}\left(\frac{z - x}{h}\right)  f^{(1)}(z)\right]_{-\infty}^{\infty} - \int_{-\infty}^\infty \frac{(-1)^s}{h^{s}}K^{(s-1)}\left(\frac{z - x}{h}\right) f^{(1)}(z)dz.\\
&=- \int_{-\infty}^\infty \frac{(-1)^s}{h^{s}}K^{(s-1)}\left(\frac{z - x}{h}\right) f^{(1)}(z)dz
\end{align*}
Repeating this $s$ times give
\begin{align*}
\E[\hat f^{(s)}(x)] &= (-1)^s \int_{-\infty}^\infty \frac{(-1)^s}{h}K\left(\frac{z - x}{h}\right) f^{(s)}(z)dz\\
&=\int_{-\infty}^\infty \frac{1}{h}K\left(\frac{z - x}{h}\right) f^{(s)}(z)dz
\end{align*}
Next, use the following change of variables: $u = \frac{z - x}{h}$, which implies $z = x + hu \implies dz = hdu$. Thus,
\begin{align}
\E[\hat f^{(s)}(x)] &= \int_{-\infty}^\infty K(u) f^{(s)}(x+hu)du \label{eq:der1}
\end{align}
The next step is to take a Taylor expansion of $f^{(s)}(x+hu)$ around $x+hu = x$, which is valid if $h \to 0$. We get
\begin{align*}
f^{(s)}(x+hu) = f^{(s)}(x) + f^{(s+1)}(x)hu + \frac{1}{2}f^{(s+2)}(x)h^2u^2 + ... + \frac{1}{P!} f^{(s+P)}(x)h^Pu^P + o(h^P).
\end{align*}
Substituting this expression back into (\ref{eq:der1}), integrating over each term, and using the fact that $\int_{-\infty}^\infty K(u)du = 1$ and the notation
\begin{align*}
\mu_\ell(K) = u^\ell K(u)
\end{align*}
gives
\begin{align*}
\E[\hat f^{(s)}(x)] &= f^{(s)}(x) + f^{(s+1)}(x)h\mu_1(K) + \frac{1}{2}f^{(s+2)}(x)h^2\mu_2(K) + ... + \frac{1}{P!} f^{(s+P)}(x)h^P\mu_P(K) + o(h^P).
\end{align*}
Finally, noting that since $K$ is a $P$-order kernel, $\mu_\ell(K) = 0$ for all $\ell < P$, gives the desired result
\begin{align}
\E[\hat f^{(s)}(x)] &= f^{(s)}(x) + \frac{1}{P!} f^{(s+P)}(x)h^P\mu_P(K) + o(h^P). \label{eq:der2}
\end{align}

Next we consider the variance of the derivative estimator.
\begin{align*}
\V[\hat f^{(s)}(x)] &= \V\left[\frac{(-1)^s}{nh^{1+s}}\sum_{i=1}^n K^{(s)}\left(\frac{X_i - x}{h}\right)\right]\\
&=\frac{1}{nh^{2+2s}}\V\left[K^{(s)}\left(\frac{X_i - x}{h}\right)\right],
\end{align*}
since $\{X_i\}$ are iid there are no covariance terms and each term has the same variance. Continuing,
\begin{align}
\V[\hat f^{(s)}(x)] &=\frac{1}{nh^{2+2s}} \left\{\E\left[K^{(s)}\left(\frac{X_i - x}{h}\right)^2\right] - \E\left[K^{(s)}\left(\frac{X_i - x}{h}\right)\right]^2 \right\} \nonumber \\
&=\frac{1}{nh^{2+2s}} \E\left[K^{(s)}\left(\frac{X_i - x}{h}\right)^2\right] - \frac{1}{n} \E\left[\frac{1}{h^{1+s}}K^{(s)}\left(\frac{X_i - x}{h}\right)\right]^2 \label{eq:der3}
\end{align}
Now, from above we know that
\begin{align*}
\E\left[\frac{1}{h^{1+s}}K^{(s)}\left(\frac{X_i - x}{h}\right)\right] & =  f^{(s)}(x) + \frac{1}{P!} f^{(s+P)}(x)h^P\mu_P(K) + o(h^P)\\
&= f^{(s)}(x) + o(1)
\end{align*}
since the remainder goes to zero as $h\to0$. Thus, the second term in (\ref{eq:der3}) is $O(\frac{1}{n})$; i.e. the same order as $1/n$. Furthermore $O(\frac{1}{n})$ is of smaller order than $O(\frac{1}{nh^{1+2s}})$ since $h \to 0$ and $n\to \infty$. Accordingly, we can write
\begin{align*}
\V[\hat f^{(s)}(x)] &=\frac{1}{nh^{2+2s}} \E\left[K^{(s)}\left(\frac{X_i - x}{h}\right)^2\right] + o\left(\frac{1}{nh^{1+2s}}\right),
\end{align*}
Thus,
\begin{align*}
\V[\hat f^{(s)}(x)] &=\frac{1}{nh^{1+2s}} \int_{-\infty}^{\infty}\frac{1}{h}K^{(s)}\left(\frac{z- x}{h}\right)^2 f(z) dz + o\left(\frac{1}{nh^{1+2s}}\right)
\end{align*}
Again we use the change of variables $u = \frac{z - x}{h}$ so that
\begin{align*}
\V[\hat f^{(s)}(x)] &=\frac{1}{nh^{1+2s}} \int_{-\infty}^{\infty}K^{(s)}(u)^2 f(x+hu) du + o\left(\frac{1}{nh^{1+2s}}\right)
\end{align*}
With the usual Taylor expansion of $f(x+hu)$ we can write
\begin{align*}
\V[\hat f^{(s)}(x)] &=\frac{1}{nh^{1+2s}} \int_{-\infty}^{\infty}K^{(s)}(u)^2 (f(x) + O(h)) du + o\left(\frac{1}{nh^{1+2s}}\right)\\
&=\frac{f(x)}{nh^{1+2s}} \int_{-\infty}^{\infty}K^{(s)}(u)^2 du + o\left(\frac{1}{nh^{1+2s}}\right)\\
&=\frac{1}{nh^{1+2s}}f(x)\vartheta_{s}(K) + o\left(\frac{1}{nh^{1+2s}}\right),
\end{align*}
where $\vartheta_{s}(K) =  \int_{-\infty}^{\infty}K^{(s)}(u)^2 du$ as required.

\newpage

\subsection{Optimal bandwidth}
We have
\begin{align*}
\text{AIMSE} [h] &= \int_{-\infty}^{\infty} \left[\left(h^P \mu_P(K) \cdot \frac{f^{(P+s)}(x)}{P!}\right)^2+\frac{1}{nh^{1+2s}}\vartheta_s(K)f(x)\right]dx\\
&= h^{2P}\left(\frac{\mu_P(K)}{P!}\right)^2\vartheta_{s+P}(f) + \frac{1}{nh^{1+2s}}\vartheta_s(K),
\end{align*}
since $f(x)$ integrates to 1 and where $\vartheta_{s+P}(f) = \int (f^{(P+s)}(x))^2dx$. Thus,
\begin{align*}
\frac{d}{dh}\text{AIMSE} [h] = 2Ph^{2P-1}\left(\frac{\mu_P(K)}{P!}\right)^2\vartheta_{s+P}(f) - (1+2s)\frac{1}{nh^{2+2s}}\vartheta_s(K)&=0\\
\implies 2Ph^{1+2P+2s}\left(\frac{\mu_P(K)}{P!}\right)^2\vartheta_{s+P}(f)  &=(1+2s)\frac{1}{n}\vartheta_s(K),
\end{align*}
which gives the optimal bandwidth
\begin{align*}
h^* = \left[\frac{1+2s}{2Pn}\left(\frac{P!}{\mu_P(K)}\right)^2\frac{\vartheta_s(K)}{\vartheta_{s+P}(f)}\right]^{\frac{1}{1+2P+2s}}.
\end{align*}
A fully data-driven method for estimating $h^*$ is cross-validation. This procedure attempts to directly estimate the mean-squared error, and then choose the bandwidth which minimizes this estimate. From the lecture notes the cross-validation bandwidth is the value $h$ which minimizes the criteria
\begin{align*}
\hat h_{CV} = \arg \min_{h} CV(h) &= \frac{1}{n^2h} \sum_{i=1}^n \sum_{j=1}^n (K*K) \left(\frac{X_i - X_j}{h}\right) - \frac{2}{n}\sum_{i=1}^n \hat f_{(i)}(X_i)
\end{align*}
where $ \hat f_{(i)}(x_i)$ is the density estimate computed without observation $X_i$.

\subsection{Monte Carlo experiment}

\textbf{(a)} First, we want to compute the theoretically optimal bandwidth for $s=0$, $n=1000$, using the Epanechnikov kernel $(P=2)$, with the following Gaussian DGP:
\begin{align*}
x_i \sim 0.5 \N(-1.5,-1.5) + 0.5\N(1,1)
\end{align*}
From Table 1 in Hansen's notes, $\mu_2(K)=1/5$ and $\vartheta(K)=3/5$ for the Epanechnikov kernel. Thus, the only other ingredient we need is $\vartheta_{2}(f) = \int [f^{(2)}(x)]^2 dx$ for the above DGP.
\begin{align*}
h^*
\end{align*}

\end{document}
