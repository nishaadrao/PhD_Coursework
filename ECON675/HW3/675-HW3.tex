\documentclass[12pt]{article}

%useful packages
\usepackage{color,soul}
\usepackage[usenames,dvipsnames,svgnames,table]{xcolor}
\usepackage{amsmath,amsthm,amscd,amssymb,bm}
\usepackage{hyperref}
\hypersetup{
    colorlinks=true,
    linkcolor=JungleGreen
}
\usepackage[utf8]{inputenc}
\usepackage[top=2cm, bottom=3cm, left=2cm, right=2cm]{geometry}
\usepackage{pgfplots}
\usepackage{enumitem}
\usepgfplotslibrary{fillbetween}
\usetikzlibrary{patterns}
\usepackage{tcolorbox}
\usepackage{centernot}
\usepackage{mathtools}
\usepackage{xcolor}

%personal definitions and commands
\newcommand{\R}{\mathbb{R}} 
\newcommand{\E}{\mathbb{E}}
\newcommand{\V}{\mathbb{V}}
\newcommand{\C}{\mathbb{C}}
\newcommand{\Prob}{\mathbb{P}}
\newcommand{\e}{\epsilon}
\newcommand\numberthis{\addtocounter{equation}{1}\tag{\theequation}} %allows numbering of single equations in align* environment
\newcommand{\mtx}[1]{\ensuremath{\bm{\mathit{#1}}}}
\newcommand{\B}{\hat{\boldsymbol{\beta}}}
\newcommand{\Cov}{\mathbb{C}\text{ov}}
\newcommand{\N}{\mathcal{N}}



\title{ECON675 -- Assignment 3}
\author{Anirudh Yadav}
\setlength\parindent{0pt}
\begin{document}

\maketitle

\setcounter{tocdepth}{2}
\tableofcontents

\section{Non-linear least squares}

\subsection{Identifiability}
This is a standard M-estimation problem. The parameter vector $\mtx{\beta}_0$ is assumed to solve the population problem
\begin{align*}
\mtx{\beta}_0 = \arg \min_{\beta \in \R^d} \E[(y_i - \mu(\mtx{x}_i'\mtx{\beta}))^2].
\end{align*}
For $\mtx{\beta}_0$ to be identified, it must be the \textit{unique} solution to the above population problem (i.e. the unique minimizer). In math, this means for all $\e>0$ and for some $\delta >0$:
\begin{align*}
\sup_{|| \beta - \beta_0 || > \e} M(\mtx{\beta}) \geq M(\mtx{\beta}_0) + \delta
\end{align*}
where $M(\mtx{\beta}) = \E[(y_i - \mu(\mtx{x}_i'\mtx{\beta}))^2]$. Of course $\mtx{\beta}_0$ can be written in closed form if $\mu(\cdot)$ is linear. In this case, we know that 
\begin{align*}
\mtx{\beta}_0= \E[\mtx{x}_i\mtx{x}_i']^{-1}\E[\mtx{x}_iy_i].
\end{align*}

\subsection{Asymptotic normality}





\end{document}
