\documentclass[12pt]{article}

%useful packages
\usepackage{color,soul}
\usepackage[usenames,dvipsnames,svgnames,table]{xcolor}
\usepackage{amsmath,amsthm,amscd,amssymb,bm}
\usepackage{hyperref}
\hypersetup{
    colorlinks=true,
    linkcolor=JungleGreen
}
\usepackage[utf8]{inputenc}
\usepackage[top=2cm, bottom=3cm, left=2cm, right=2cm]{geometry}
\usepackage{pgfplots}
\usepackage{enumitem}
\usepgfplotslibrary{fillbetween}
\usetikzlibrary{patterns}
\usepackage{tcolorbox}
\usepackage{centernot}
\usepackage{mathtools}
\usepackage{xcolor}

% Packages to create table 1
\usepackage{pdflscape}
\usepackage{subfig}
\usepackage{graphicx}


%personal definitions and commands
\newcommand{\R}{\mathbb{R}} 
\newcommand{\E}{\mathbb{E}}
\newcommand{\V}{\mathbb{V}}
\newcommand{\C}{\mathbb{C}}
\newcommand{\Prob}{\mathbb{P}}
\newcommand{\e}{\epsilon}
\newcommand\numberthis{\addtocounter{equation}{1}\tag{\theequation}} %allows numbering of single equations in align* environment
\newcommand{\mtx}[1]{\ensuremath{\bm{\mathit{#1}}}}
\newcommand{\B}{\hat{\boldsymbol{\beta}}}
\newcommand{\Cov}{\mathbb{C}\text{ov}}
\newcommand{\N}{\mathcal{N}}



\title{ECON675 -- Assignment 4}
\author{Anirudh Yadav}
\setlength\parindent{0pt}
\begin{document}

\maketitle

\setcounter{tocdepth}{2}
\tableofcontents

\newpage

\section{Estimating equations}

\subsection{Moment conditions}
The goal of this question is to show that the four given functions are valid moment conditions for the parameter $\theta_t(g)$. That is, we want to show that 
\begin{align*}
\E[\psi_{\texttt{f},t}(\mtx{Z}_i; \theta_t(g))] = 0,
\end{align*}
for each $\texttt{f} \in \{\texttt{IPW},\texttt{RI1},\texttt{RI2}, \texttt{DR}\}$. Note that in the derivations below I invoke LIE a lot without specifically mentioning it.\\

Start with the inverse probability weighting function
\begin{align*}
\E[\psi_{\texttt{IPW},t}(\mtx{Z}_i; \theta_t(g))] &=\E\left[\frac{D_i(t)\cdot g(Y_i(t))}{p_t(\mtx{X}_i)}\right] - \theta_t(g)\\
&=\E\left[\E\left[\frac{D_i(t)\cdot g(Y_i(t))}{p_t(\mtx{X}_i)}| \mtx{X}_i\right]\right] - \theta_t(g)\\
&=\E\left[\frac{1}{p_t(\mtx{X}_i)}\E\left[D_i(t)|\mtx{X}_i\right]\E\left[ g(Y_i(t))| \mtx{X}_i\right]\right] - \theta_t(g)
\end{align*}
Now, 
\begin{align*}
\E\left[D_i(t)|\mtx{X}_i\right] & = \Pr[D_i(t)=1|\mtx{X}_i] = \Pr[T_i = t|\mtx{X}_i] = p_t(\mtx{X}_i).
\end{align*}
Thus,
\begin{align*}
\E[\psi_{\texttt{IPW},t}(\mtx{Z}_i; \theta_t(g))] &= \E\left[\E\left[ g(Y_i(t))| \mtx{X}_i\right]\right] - \theta_t(g)\\
&=\E[g(Y_i(t))] - \theta_t(g)\\
&=0.
\end{align*}
Next, consider
\begin{align*}
\E[\psi_{\texttt{RI1},t}(\mtx{Z}_i; \theta_t(g))] &= \E[e_t(g;\mtx{X}_i)] - \theta_t(g)\\
&=\E[\E[g(Y_i(t)|\mtx{X}_i]] - \theta_t(g)\\
&=\E[g(Y_i(t)] -  \theta_t(g)\\
&=0.
\end{align*}
And,
\begin{align*}
\E[\psi_{\texttt{RI2},t}(\mtx{Z}_i; \theta_t(g))] &= \E\left[\frac{D_i(t)\cdot e_t(g;\mtx{X}_i)}{p_t(\mtx{X}_i)}\right]  - \theta_t(g)\\
&=\E\left[\E\left[\frac{D_i(t)\cdot e_t(g;\mtx{X}_i)}{p_t(\mtx{X}_i)}| \mtx{X}_i\right]\right] - \theta_t(g)\\
&=\E\left[\E\left[e_t(g;\mtx{X}_i)| \mtx{X}_i\right]\right] - \theta_t(g)\\
&= \E[e_t(g;\mtx{X}_i)] - \theta_t(g)\\
&=0.
\end{align*}
Finally, consider the doubly robust function
\begin{align*}
\E[\psi_{\texttt{DR},t}(\mtx{Z}_i; \theta_t(g))] &= \E\left[\frac{D_i(t)\cdot g(Y_i(t))}{p_t(\mtx{X}_i)}\right] - \theta_t(g) - \E\left[\frac{e_t(g;\mtx{X}_i)}{p_t(\mtx{X}_i)}(D_i(t)-p_t(\mtx{X}_i))\right].
\end{align*}
Using the IPW result above, we know that the first two terms cancel each other out, so that
\begin{align*}
\E[\psi_{\texttt{DR},t}(\mtx{Z}_i; \theta_t(g))] &=- \E\left[\frac{e_t(g;\mtx{X}_i)}{p_t(\mtx{X}_i)}(D_i(t)-p_t(\mtx{X}_i))\right]\\
&=- \E\left[\frac{e_t(g;\mtx{X}_i)D_i(t)}{p_t(\mtx{X}_i)}\right] + \E[e_t(g;\mtx{X}_i)]\\
&= -\theta_t(g) + \theta_t(g)\\
&=0.
\end{align*}
So each of the four functions is a valid moment condition for $\theta_t(g)$.

\subsection{Plug-in estimators}
The plug-in IPW estimator is 
\begin{align*}
\hat{\theta}_{\texttt{IPW},t}(g) = \frac{1}{n}\sum_{i=1}^n\frac{D_i(t)g(Y_i)}{\hat p_t(\mtx{X}_i)},
\end{align*}
where $\hat p_t(\mtx{X}_i)$ is the estimated propensity score. Note that since there are multiple treatment levels, the estimated propensity score would have to be computed using a suitable discrete choice model. For instance, $\hat p_t(\mtx{X}_i)$ could be estimated using a multinomial logit model.\\

The plug-in projection (or regression imputation) estimator is
\begin{align*}
\hat{\theta}_{\texttt{RI1},t}(g) = \hat{\E}[e_t(g;\mtx{X}_i)] =  \frac{1}{n}\sum_{i=1}^n \hat{\E}[g(Y_i(t))|\mtx{X}_i] &=  \frac{1}{n}\sum_{i=1}^n \hat{\E}[g(Y_i(t))|\mtx{X}_i, D_i(t)=1] \\
&= \frac{1}{n}\sum_{i=1}^n \hat{\E}[g(Y_i)|\mtx{X}_i, D_i(t)=1],
\end{align*}
where the second last equality uses the ignorability assumption. We need to make a choice about how to estimate the conditional expectation term. I think we could use NLS, or possibly a nonparametric method like kernel regression. To ease notation, let $\widehat{\mu}_t(\mtx{X}_i)$ be the parametric or nonparametric estimate of $\E[g(Y_i)|\mtx{X}_i, D_i(t)=1]$. Then, the projection estimator is
\begin{align*}
\hat{\theta}_{\texttt{RI1},t}(g) = \frac{1}{n}\sum_{i=1}^n \widehat{\mu}_t(\mtx{X}_i)
\end{align*}

The plug-in `hybrid' imputation estimator
\begin{align*}
\hat{\theta}_{\texttt{RI2},t}(g) &=  \frac{1}{n}\sum_{i=1}^n\frac{D_i(t)\widehat{\mu}_t(\mtx{X}_i)}{\hat p_t(\mtx{X}_i)}. 
\end{align*}
Finally, the plug-in doubly robust estimator is given by
\begin{align*}
\hat{\theta}_{\texttt{DR},t}(g) &= \frac{1}{n}\sum_{i=1}^n\frac{D_i(t)g(Y_i)}{\hat p_t(\mtx{X}_i)}-\frac{1}{n}\sum_{i=1}^n\frac{\widehat{\mu}_t(\mtx{X}_i)}{\hat p_t(\mtx{X}_i)}(D_i(t) - \hat p_t(\mtx{X}_i))\\
&=\frac{1}{n}\sum_{i=1}^n \left(\frac{D_i(t)(g(Y_i) - \widehat{\mu}_t(\mtx{X}_i))}{\hat p_t(\mtx{X}_i)} + \widehat{\mu}_t(\mtx{X}_i) \right).
\end{align*}
As discussed in Abadie and Catteneo (2018), the relative performance of the above estimators depends on the features of the data generating process. In finite samples, IPW estimators become unstable when the propensity score approaches zero or one and regression imputation estimators may suffer from extrapolation biases. Doubly robust estimators include safeguards against bias caused by misspecification but impose additional specification choices that may affect the resulting estimate.

\subsection{Estimating the variance of potential outcomes}


\newpage
\section{Estimating average treatment effects}

Some notes on my progress so far:
\begin{itemize}
\item I've computed most of the RI, IPW and DR estimates for ATE in \verb|R| using the \verb|CausalGAM| package. Note that I specified a logit model to compute propensity scores. This package also computes bootstrap and (estimated) asymptotic SEs. 
\item The logit regression for the PSID sample using covariate set A does not converge in \verb|STATA| or using \verb|CausalGAM|. (Weirdly, the logit does converge if I compute it manually in \verb|R|).
\item As far as I can tell, the \verb|CausalGAM| package does not allow easy computation of ATTs, so I think I will have to use \verb|STATA| for these, which is a bit of a shame.
\item On an initial run, it looks like \verb|STATA| spits out the same estimates for IPW and DR, which is weird because \verb|CausalGAM| does not (and the \verb|CausalGAM| estimates of the IPW estimator match \verb|STATA|'s).
\end{itemize}


\begin{landscape}
\begin{table}
\centering
\caption{Estimation and Inference on ATE and ATT}\label{tab:tableQ2}
\vspace{-.1in}\resizebox{\columnwidth}{!}{
\subfloat[][ATE]{\input{teffectsATEq.txt}}\quad
\subfloat[][ATT]{\input{teffectsATTq.txt}}
}
\end{table}
\end{landscape}

\newpage

\end{document}
