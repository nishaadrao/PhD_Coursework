\documentclass[12pt]{article}

%useful packages
\usepackage{color,soul}
\usepackage[usenames,dvipsnames,svgnames,table]{xcolor}
\usepackage{amsmath,amsthm,amscd,amssymb,bm}
\usepackage{hyperref}
\hypersetup{
    colorlinks=true,
    linkcolor=JungleGreen
}
\usepackage[utf8]{inputenc}
\usepackage[top=2cm, bottom=3cm, left=2cm, right=2cm]{geometry}
\usepackage{pgfplots}
\usepackage{enumitem}
\usepgfplotslibrary{fillbetween}
\usetikzlibrary{patterns}
\usepackage{tcolorbox}
\usepackage{centernot}
\usepackage{mathtools}
\usepackage{xcolor}

%personal definitions and commands
\newcommand{\R}{\mathbb{R}} 
\newcommand{\E}{\mathbb{E}}
\newcommand{\V}{\mathbb{V}}
\newcommand{\C}{\mathbb{C}}
\newcommand{\Prob}{\mathbb{P}}
\newcommand{\e}{\epsilon}
\newcommand\numberthis{\addtocounter{equation}{1}\tag{\theequation}} %allows numbering of single equations in align* environment
\newcommand{\mtx}[1]{\ensuremath{\bm{\mathit{#1}}}}
\newcommand{\B}{\hat{\boldsymbol{\beta}}}
\newcommand{\Cov}{\mathbb{C}\text{ov}}
\newcommand{\N}{\mathcal{N}}



\title{ECON675 -- Assignment 4}
\author{Anirudh Yadav}
\setlength\parindent{0pt}
\begin{document}

\maketitle

\setcounter{tocdepth}{2}
\tableofcontents

\newpage

\section{Estimating equations}

\subsection{Moment conditions}
The goal of this question is to show that the four given functions are valid moment conditions for the parameter $\theta_t(g)$. That is, we want to show that 
\begin{align*}
\E[\psi_{\texttt{f},t}(\mtx{Z}_i; \theta_t(g))] = 0,
\end{align*}
for each $\texttt{f} \in \{\texttt{IPW},\texttt{RI1},\texttt{RI2}, \texttt{DR}\}$. Note that in the derivations below I invoke LIE a lot without specifically mentioning it.\\

Start with the inverse probability weighting function
\begin{align*}
\E[\psi_{\texttt{IPW},t}(\mtx{Z}_i; \theta_t(g))] &=\E\left[\frac{D_i(t)\cdot g(Y_i(t))}{p_t(\mtx{X}_i)}\right] - \theta_t(g)\\
&=\E\left[\E\left[\frac{D_i(t)\cdot g(Y_i(t))}{p_t(\mtx{X}_i)}| \mtx{X}_i\right]\right] - \theta_t(g)\\
&=\E\left[\frac{1}{p_t(\mtx{X}_i)}\E\left[D_i(t)|\mtx{X}_i\right]\E\left[ g(Y_i(t))| \mtx{X}_i\right]\right] - \theta_t(g)
\end{align*}
Now, 
\begin{align*}
\E\left[D_i(t)|\mtx{X}_i\right] & = \Pr[D_i(t)=1|\mtx{X}_i] = \Pr[T_i = t|\mtx{X}_i] = p_t(\mtx{X}_i).
\end{align*}
Thus,
\begin{align*}
\E[\psi_{\texttt{IPW},t}(\mtx{Z}_i; \theta_t(g))] &= \E\left[\E\left[ g(Y_i(t))| \mtx{X}_i\right]\right] - \theta_t(g)\\
&=\E[g(Y_i(t))] - \theta_t(g)\\
&=0.
\end{align*}
Next, consider
\begin{align*}
\E[\psi_{\texttt{RI1},t}(\mtx{Z}_i; \theta_t(g))] &= \E[e_t(g;\mtx{X}_i)] - \theta_t(g)\\
&=\E[\E[g(Y_i(t)|\mtx{X}_i]] - \theta_t(g)\\
&=\E[g(Y_i(t)] -  \theta_t(g)\\
&=0.
\end{align*}
And,
\begin{align*}
\E[\psi_{\texttt{RI2},t}(\mtx{Z}_i; \theta_t(g))] &= \E\left[\frac{D_i(t)\cdot e_t(g;\mtx{X}_i)}{p_t(\mtx{X}_i)}\right]  - \theta_t(g)\\
&=\E\left[\E\left[\frac{D_i(t)\cdot e_t(g;\mtx{X}_i)}{p_t(\mtx{X}_i)}| \mtx{X}_i\right]\right] - \theta_t(g)\\
&=\E\left[\E\left[e_t(g;\mtx{X}_i)| \mtx{X}_i\right]\right] - \theta_t(g)\\
&= \E[e_t(g;\mtx{X}_i)] - \theta_t(g)\\
&=0.
\end{align*}
Finally, consider the doubly robust function
\begin{align*}
\E[\psi_{\texttt{DR},t}(\mtx{Z}_i; \theta_t(g))] &= \E\left[\frac{D_i(t)\cdot g(Y_i(t))}{p_t(\mtx{X}_i)}\right] - \theta_t(g) - \E\left[\frac{e_t(g;\mtx{X}_i)}{p_t(\mtx{X}_i)}(D_i(t)-p_t(\mtx{X}_i))\right].
\end{align*}
Using the IPW result above, we know that the first two terms cancel each other out, so that
\begin{align*}
\E[\psi_{\texttt{DR},t}(\mtx{Z}_i; \theta_t(g))] &=- \E\left[\frac{e_t(g;\mtx{X}_i)}{p_t(\mtx{X}_i)}(D_i(t)-p_t(\mtx{X}_i))\right]\\
&=- \E\left[\frac{e_t(g;\mtx{X}_i)D_i(t)}{p_t(\mtx{X}_i)}\right] + \E[e_t(g;\mtx{X}_i)]\\
&= -\theta_t(g) + \theta_t(g)\\
&=0.
\end{align*}
So each of the four functions is a valid moment condition for $\theta_t(g)$.

\end{document}
